\documentclass[a5j,11pt,uplatex,dvipdfmx]{jsbook}
%
%%%%========ここからプリアンブル=============%%%%%%%%%%%%
%
%%%%%%パッケージの取り込み%%%%%%%%%%%%
%
\usepackage{bxpapersize} % 読み込みだけでOK 紙サイズをなんとかしてくれる
\usepackage{amsmath,amssymb} % 数式用
\usepackage[dvipdfmx]{graphicx} % 画像の挿入
\usepackage{amsthm} % 定理環境
\usepackage{wrapfig} % 図を回り込ませるやつ
\usepackage{comment} % 複数行のコメント用
\usepackage{enumerate}
\usepackage{braket} % カッコの拡張 \Set など
\usepackage{okumacro} % ルビを振りたい
\usepackage{bussproofs} % 証明図用
 \renewcommand{\defaultHypSeparation}{\hskip .1in} % 導出図の仮定同士の感覚
\usepackage{longtable} % 長い表
\usepackage{mathrsfs} % 花文字
\usepackage{framed} % 枠囲み 
\usepackage{setspace} % 行間の調整
% \usepackage{makeidx} % 索引
 \usepackage{index} % 複数の索引
\usepackage[subrefformat=parens]{subcaption} % 複数の図
 \captionsetup{compatibility=false} % おまじない
\allowdisplaybreaks % 数式中での改ページを許可
 %
 %
 %%%%%%%%%===============索引================================%%%%%%%%%%%%
 % \makeindex % 索引の作成
  \newindex{nidx}{aidx}{aind}{人名索引} % 人名索引
  \newindex{widx}{bidx}{bind}{用語索引} % フツーの索引
 %%%%%%%%%%%%%%%%%%%%%%%%%%%%%%%%%%%%%%%%%%%%
%%%%%=============演算子の定義=======================%%%%%%%%%%%%%%%%%%%
 \newcommand{\id}{\mathop{\mathrm{id}}\nolimits} % \id で恒等写像
 \newcommand{\Map}{\mathop{\mathrm{Map}}\limits} % \Map で写像全体の集合
%
%
%
%%%%%%==========定理環境のカスタマイズ===============%%%%%%%%%
%
\newtheoremstyle{mystyle} % 定理環境のスタイルを作成
  {} % space above
  {} % space below
  {\normalfont} % body font 
  {} % indent amount
  {\bfseries} % theorem head font
  {} % punctuation after theorem head
  {10pt} % space after theorem head
  {\thmname{#1}\thmnumber{#2}\thmnote{\textbf{\hspace{1pt}(#3)}}} % theorem head spec
\theoremstyle{mystyle} % 作成したスタイルを使用
%%%==========これで定理環境のあとのドットが消えるらしい================%%%%%%%%%%%
%
%  \theorembodyfont{\normalfont} % 定理環境中のフォントをフツーのものに
%% 
    \newtheorem*{dfn}{定義} % \begin{dfn} で定義を書く
    \newtheorem{thm}{定理}[chapter] % \begin{thm} で定理を書く
    \newtheorem{ex}{例}[chapter] % \begin{ex} で例を出す
    \newtheorem{que}{問}[chapter] % \begin{que} で問を出す
    \newtheorem*{axiom}{公理} % \begin{axiom} で公理
    \newtheorem{lemma}{補題}[chapter] % \begin{lemma} で補題
    \newtheorem*{coro}{系}  % \begin{coro} で系
%%
%%%%%%%%%=========定理環境のカスタマイズ終わり===========%%%%%%%%%%
%
%%%%%================proof環境のカスタマイズ========================%%%%%%%%%%%
  \makeatletter
    \renewenvironment{proof}[1][\proofname]{\par
      \pushQED{\qed}%
        \normalfont \topsep6\p@\@plus6\p@\relax
          \trivlist
            \item[\hskip\labelsep
                      \itshape
                      %    #1\@addpunct{.}]\ignorespaces% DELETED
                          #1]\ignorespaces% ADDED
                          }{%
                              \popQED\endtrivlist\@endpefalse
                              }
                              \makeatother
%                       ドットが消える
    \renewcommand{\proofname}{\textbf{[証明]}}
%
%
%%%%%%%%%===節番号の仕様変更========%%%%%%%%
%
 \renewcommand{\thesection}{{\S} \arabic{chapter}.\arabic{section}}
%
%%%%%%%%============ 節番号にセクション記号が使われる============%%%%%%%
%
 \renewcommand{\thefigure}{\arabic{chapter}.\arabic{figure}}
 \renewcommand{\thetable}{\arabic{chapter}.\arabic{table}}
%
%
%%%%%%%%%%%%%%%%%=============定理環境の番号の仕様変更================%%%%%%%%%%%%%
 %
 %%% 番号からセクション記号を消したい %%%%%%
   % dfn は番号なしなので不要
   \renewcommand{\thethm}{\arabic{chapter}.\arabic{thm}}
   \renewcommand{\theex}{\arabic{chapter}.\arabic{ex}}
   \renewcommand{\theque}{\arabic{chapter}.\arabic{que}}
  % \renewcommand{\theaxiom}{\arabic{chapter}.\arabic{axiom}}
   \renewcommand{\thelemma}{\arabic{chapter}.\arabic{lemma}}
  %
%%%%%%%%%%%%%%%%================定理環境の番号の仕様変更終わり=============%%%%%%%%%%%
%
%
\newcommand{\refque}[1]{問\ref{#1}}
%   
%   
%%%%%%============enumerate環境の番号=====================================%%%%%%%%%%%
%
 % \renewcommand{\labelenumi}{\textbf{問\arabic{chapter}.\arabic{section}.\arabic{enumi}.}} 
   \renewcommand{\labelenumi}{(\arabic{enumi})}
%
%
%%%%%%%%%%%%%%%%%%%%%%%%%%%%%============数式番号の仕様変更===============%%%%%%%%%%%%%%%%%%
\makeatletter
 \renewcommand{\theequation}
  {\arabic{chapter}.\arabic{section}.\arabic{equation}}
   \@addtoreset{equation}{section} % 数式番号
\makeatother
%%%%%%%%%%%%%%%%=============================================%%%%%%%%%%%%%%%%%%%%%%%%%%%%%%%%
   %
   %
%%%%%%%%%%%%%%%%%%==============数式の見栄えを良くする====================%%%%%%%%%%%%
%
%
%
   \renewcommand{\frac}[2]{{{\, #1 \,}\over{\, #2 \,}}} % \fracコマンドの棒を長くする
%
%
%%%%%%%%%%%%%==================================================================%%%%%%%%%%%%%%%% 
 %
\begin{document} % 本文開始
%
%
%%%%%%%=====タイトルページ==========%%%%%%%%
%
 \begin{titlepage} % タイトルページ
   \title{記号論理超入門}
  \author{高知工科大学環境理工学群4年 野口 匠}
  \date{2018/7/30}
 \maketitle
 \thispagestyle{empty}
 \end{titlepage}
%
%%%=======序文と目次===========%%%%%%%%%%
%
\frontmatter % 序文
 \chapter{はじめに}
 この資料は,これから数学を本格的に学ぼうとする高校生,大学生を対象に,
 数学全体の基礎となっている「論理」や「推論」の感覚をつかんでもらうために書いたものである.
 
 本格的に数学を学ぼうとしたとき,
 まず第一に数学的な議論のフォーマットに体を慣らさなくてはならない.
 数学的な議論のフォーマットは,
 実際に数学を学ぶことを通して体得するのが普通である.
 そうして数学を学んでいくと,
 数学で使われる「推論」のなかには,直感的に明らかとは言い難いものが
 そこそこ多くあることに気がつくはずである.
 \index[nidx]{De Morgan@De Morgan(ド・モルガン)}
 De Morganの法則や対偶証明法,あるいは背理法などが挙げられるだろう.
 初めてそれらの論法が提示された際,
 ちょっとモヤモヤした気分になった人は私だけではあるまい.
 
 この資料では,まずは意味論的に論理記号を導入し,
 その後,自然演繹と呼ばれる演繹体系に基づいて
 これらの論法を正当化することを試みる.
 ただし,ここでは自然演繹そのものではなく,
 シークエント計算と呼ばれる手法を利用した自然演繹の体系を紹介する.
 個人的な意見ではあるが,自然演繹そのものよりもシークエント計算
 のほうがとっつきやすい.
 とはいえ,慣れるまではかなり苦労すると思われるから,
 根気よく学んでもらいたい.
 ページ数自体はあまり多くはないが,
 読み通すまでにはかなり時間がかかると思われる.

 なお,この資料は私1人で書いたものであり,
 校正も私自身が行っている.
 従って,普通の本よりも間違いや誤植が存在する可能性が高いということに
 留意されたい.
 もしも間違いや誤植を見つけた際には,以下に連絡していただけるとありがたい.

 \begin{itemize}
   \item TwitterID:@NOGUTAKULab
   \item E-mail:nogutakulab@gmail.com
 \end{itemize}

  
\begin{flushright}
  2018年1月 著者
\end{flushright}

 更新履歴
 \begin{itemize}
   \item 2018/1/10 公開
   \item 2018/1/11 誤植の修正,連絡先の追加
   \item 2018/4/5 「現代数学への展望」を追加,演習問題の追加,その他不適切な言い回しの修正
   \item 2018/7/30 レイアウトの修正,英訳の訂正,数学的構造の解説の追加など
 \end{itemize}
  
 % 序文
  \begin{table}[htbp]
   \centering
   \captionsetup{labelformat=empty,labelsep=none} % 評判号を非表示に
   \caption{ギリシャ文字一覧}
   \scalebox{0.9}{
   \begin{tabular}{ccc}
     \hline 
     読み & 大文字 & 小文字 \\ \hline 
     アルファ & $A$ & $\alpha$ \\
     ベータ & $B$ & $\beta$ \\
     ガンマ & $\Gamma$ & $\gamma$ \\
     デルタ & $\Delta$ & $\delta$ \\
     イプシロン,エプシロン & $E$ & $\epsilon , \, \varepsilon$ \\
     ゼータ,ツェータ & $Z$ & $\zeta $ \\
     イータ,エータ & $H$ & $\eta$ \\
     シータ,テータ & $\Theta$ & $\theta , \, \vartheta$ \\
     イオタ & $I$ & $\iota$  \\
     カッパ & $K$ & $\kappa$ \\
     ラムダ & $\Lambda$ & $\lambda$ \\
     ミュー & $M$ & $\mu$  \\
     ニュー & $N$ & $\nu$ \\
     オミクロン & $O$ & $o$ \\
     グザイ,クシー & $\Xi$ & $\xi$ \\
     パイ & $\Pi$ & $\pi, \, \varpi$ \\
     ロー & $P$ & $\rho , \, \varrho$ \\
     シグマ & $\Sigma$ & $\sigma , \, \varsigma$ \\
     タウ & $T$ & $\tau$ \\
     ユプシロン,ウプシロン & $\Upsilon$ & $\upsilon$ \\
     ファイ & $\Phi$ & $\phi , \, \varphi$ \\
     カイ,キー & $X$ & $\chi$ \\
     プサイ,プシー & $\Psi$ & $\psi$ \\
     オメガ & $\Omega$ & $\omega$ \\ \hline
   \end{tabular}
 }
 \end{table} 

 \begin{table}[htbp] 
   \centering
   \captionsetup{labelformat=empty,labelsep=none} % 表番号を非表示
   \caption{花文字一覧}
   %
   \begin{tabular}{cc|cc}
     \hline
     大文字 & 対応する大文字 & 大文字 & 対応する大文字 \\ \hline
     $\mathscr{A}$ & $A$ & $\mathscr{N}$ & $N$ \\
     $\mathscr{B}$ & $B$ & $\mathscr{O}$ & $O$ \\
     $\mathscr{C}$ & $C$ & $\mathscr{P}$ & $P$ \\
     $\mathscr{D}$ & $D$ & $\mathscr{Q}$ & $Q$ \\
     $\mathscr{E}$ & $E$ & $\mathscr{R}$ & $R$ \\
     $\mathscr{F}$ & $F$ & $\mathscr{S}$ & $S$ \\
     $\mathscr{G}$ & $G$ & $\mathscr{T}$ & $T$ \\
     $\mathscr{H}$ & $H$ & $\mathscr{U}$ & $U$ \\
     $\mathscr{I}$ & $I$ & $\mathscr{V}$ & $V$ \\
     $\mathscr{J}$ & $J$ & $\mathscr{W}$ & $W$ \\
     $\mathscr{K}$ & $K$ & $\mathscr{X}$ & $X$ \\
     $\mathscr{L}$ & $L$ & $\mathscr{Y}$ & $Y$ \\
     $\mathscr{M}$ & $M$ & $\mathscr{Z}$ & $Z$ \\ \hline
   \end{tabular}
 \end{table}
%
%
%
\begin{comment}
 \begin{table}[htbp]
   \centering
   \captionsetup{labelformat=empty,labelsep=none} % 表番号を非表示
   \caption{ドイツ文字一覧}
   \scalebox{0.86}{
     \begin{tabular}{cccc}
       \hline
       読み方 & 大文字 & 小文字 & 対応する英字 \\ \hline
       アー & $\mathfrak{A}$ & $\mathfrak{a}$ & $a$ \\
       ベー & $\mathfrak{B}$ & $\mathfrak{b}$ & $b$ \\
       ツェー & $\mathfrak{C}$ & $\mathfrak{c}$ & $c$ \\
       デー & $\mathfrak{D}$ & $\mathfrak{d}$ & $d$ \\
       エー & $\mathfrak{E}$ & $\mathfrak{e}$ & $e$ \\
       エフ & $\mathfrak{F}$ & $\mathfrak{f}$ & $f$ \\
       ゲー & $\mathfrak{G}$ & $\mathfrak{g}$ & $g$ \\
       ハー & $\mathfrak{H}$ & $\mathfrak{h}$ & $h$ \\
       イー & $\mathfrak{I}$ & $\mathfrak{i}$ & $i$ \\
       ヨット,ヤット & $\mathfrak{J}$ & $\mathfrak{j}$ & $j$ \\
       カー & $\mathfrak{K}$ & $\mathfrak{k}$ & $k$ \\
       エル & $\mathfrak{L}$ & $\mathfrak{l}$ & $l$ \\
       エム & $\mathfrak{M}$ & $\mathfrak{m}$ & $m$ \\
       エヌ & $\mathfrak{N}$ & $\mathfrak{n}$ & $n$ \\
       オー & $\mathfrak{O}$ & $\mathfrak{o}$ & $o$ \\
       ペー & $\mathfrak{P}$ & $\mathfrak{p}$ & $p$ \\
       クー & $\mathfrak{Q}$ & $\mathfrak{q}$ & $q$ \\
       エール,エア & $\mathfrak{R}$ & $\mathfrak{r}$ & $r$ \\
       エス & $\mathfrak{S}$ & $\mathfrak{s}$ & $s$ \\
       テー & $\mathfrak{T}$ & $\mathfrak{t}$ & $t$ \\
       ウー & $\mathfrak{U}$ & $\mathfrak{u}$ & $u$ \\
       ファウ & $\mathfrak{V}$ & $\mathfrak{v}$ & $v$ \\
       ヴェー & $\mathfrak{W}$ & $\mathfrak{w}$ & $w$ \\
       イクス & $\mathfrak{X}$ & $\mathfrak{x}$ & $x$ \\
       エプシロン & $\mathfrak{Y}$ & $\mathfrak{y}$ & $y$ \\
       ツェット & $\mathfrak{Z}$ & $\mathfrak{z}$ & $z$ \\ \hline
     \end{tabular}
   }
 \end{table}
%

 \begin{table}[htbp]
   \centering
   \captionsetup{labelformat=empty,labelsep=none} % 表番号を非表示
   \caption{筆記体一覧}
   \scalebox{0.95}{
   \begin{tabular}{cc|cc}
     \hline
     大文字 & 対応する大文字 & 大文字 & 対応する大文字 \\ \hline
     $\mathcal{A}$ & $A$ & $\mathcal{N}$ & $N$ \\
     $\mathcal{B}$ & $B$ & $\mathcal{O}$ & $O$ \\
     $\mathcal{C}$ & $C$ & $\mathcal{P}$ & $P$ \\
     $\mathcal{D}$ & $D$ & $\mathcal{Q}$ & $Q$ \\
     $\mathcal{E}$ & $E$ & $\mathcal{R}$ & $R$ \\
     $\mathcal{F}$ & $F$ & $\mathcal{S}$ & $S$ \\
     $\mathcal{G}$ & $G$ & $\mathcal{T}$ & $T$ \\
     $\mathcal{H}$ & $H$ & $\mathcal{U}$ & $U$ \\
     $\mathcal{I}$ & $I$ & $\mathcal{V}$ & $V$ \\
     $\mathcal{J}$ & $J$ & $\mathcal{W}$ & $W$ \\
     $\mathcal{K}$ & $K$ & $\mathcal{X}$ & $X$ \\
     $\mathcal{L}$ & $L$ & $\mathcal{Y}$ & $Y$ \\
     $\mathcal{M}$ & $M$ & $\mathcal{Z}$ & $Z$ \\ \hline
   \end{tabular}
 }
 \end{table}
\end{comment}
 \begin{table}[p]
   \centering
   \captionsetup{labelformat=empty,labelsep=none} % 表番号を非表示
   \caption{よく使う集合}
   \begin{tabular}{cc}
     \hline
     記号 & 意味 \\ \hline
     $\mathbb{N}$ & 自然数全体の集合 \\
     $\mathbb{Z}$ & 整数全体の集合 \\
     $\mathbb{Q}$ & 有理数全体の集合 \\
     $\mathbb{R}$ & 実数全体の集合 \\
     $\mathbb{C}$ & 複素数全体の集合 \\ \hline
   \end{tabular}
 \end{table}






 % ギリシャ文字とか黒板太字の集合の表記とか
 \tableofcontents % 目次
%
%%%%%========本文開始============%%%%%%%%%%%
%
\mainmatter % 本文スタート
 \chapter{現代数学への展望}
\label{chp:gensuu}
 この資料を手に取った読者の中には初めて数学を本格的に学ぼうとする人も多いだろう.
 そのような人の中には,現代数学は複雑怪奇でよくわからない世界という
 イメージを持っている人も多いと思う.
 特に,「公理とは,絶対普遍の真理である」などと考えている人も多いのではないだろうか.

 この章では,そうした「公理」や「定義」などの数学を学ぶ上では絶対に知っておかなければならない
 基本的な用語に関してざっくりと解説する.
 後半では数学的構造について解説し,位相空間論や群論などの
 現代数学の基礎となっている理論をより深く理解するための橋渡しを試みる.

 この章では,他の分野(特に代数学)で
 学ぶ事項のうちで既知としたものがある.
 この章を読むときには,
 実際に数学を学びながら暇をつぶす感覚でこの章を読むとよい.

 \newpage

 \section{幾何学の歴史と公理論}
 \label{sec:kikakouri}
  公理主義的な現代数学の価値観は,幾何学とともに形成されていったといっても過言ではない.
  ここでは,幾何学の歴史をおおざっぱに見ていこう.

 \paragraph{幾何学の起源}
  幾何学は,紀元前に古代エジプトにおいて測量の必要性から生まれた.
  しかし,そこでの「幾何学」は現代のように演繹的証明を積み重ねていくものではなかった.
  それは幾何学だけではない.当時の古代エジプトでは十進法や単位分数,
  辺の長さの比が$3:4:5$の直角三角形などが発見されていたが,
  その特徴は「具体的な例題の解法の羅列」であった.

  また,紀元前の幾何学といえば古代バビロニアも見逃せない.
  古代バビロニアでは六十進法が主として用いられた.
  ほかにも,2次方程式の解法やピタゴラスの定理についての研究もなされていたようである.
  しかし,その特徴は古代エジプトと同じく「具体的な例題の解法の羅列」であった.

 \paragraph{古代ギリシャの数学}
  古代エジプトと古代バビロニアでの幾何学に共通するのは
  「具体的な例題の解法の羅列」である.
  すなわち,古代エジプトと古代バビロニアで培われていた知識はどちらも
  実用的なものであったと考えることができる.
  ところが,古代エジプトと古代バビロニアで培われていた知識は古代ギリシャに渡り,
  そこで大きな変貌を遂げることとなった.
  それは,実用的知識をある理論体系にまとめあげ,そこから得られた結果を
  再び実用的な問題に当てはめるというものである.
  つまり,古代ギリシャでは「具体的な対象から推測される
  一般化された主張に証明を与え,それを再び具体的な対象に適用する」
  という形式で幾何学が展開されていったのである.

  \index[nidx]{Thales@Thales(ターレス)}
  Thales(624 B.C.頃--547 B.C.頃)はギリシャ哲学の祖とも言われており,
  上記のような考えのもと,さまざまな主張を一般化し,その証明を与えていった.
  以下にThalesの発見と言われていることがらをいくつか挙げておこう.
  \begin{itemize}
    \item 円は直径により2等分される.
    \item 二等辺三角形の両底角は相等しい.
    \item 2直線が交わるとき,その対頂角は相等しい.
    \item 2つの三角形において,一辺とその両端の角がそれぞれ等しければ,
      2つの三角形は合同である.
    \item 1つの円において,直径に対する円周角はつねに直角である.
  \end{itemize}

 \paragraph{Euclidの原論}
  古代ギリシャにおいては幾何学の研究は以後も続けられていたが,
  \index[nidx]{Euclid@Euclid(ユークリッド)}
  古代ギリシャの数学者Euclid(300 B.C.頃)の著書『原論』により,
  それまでに得られていた数学研究の成果が演繹的体系としてまとめ上げられた.

  『原論』では,まずこれから用いる「点」や「直線」,あるいは「面」などの
  言葉の定義がなされており,
  ついで幾何学を展開するときの基礎となる5つの公準(要請)と
  数学全体において基礎となる公理(共通概念)が述べられている.
  その後,用意した公理と公準に基づき,演繹的論証を積み重ねることによって
  当時知られていたいくつもの定理が証明されている.
  しかし,『原論』における「公理」とは「一般に共通する普遍の真理」であり,
  「公準」とは「幾何学における要請」であったことに注意しておかなければならない
  \footnote{ここでの「公準」は「幾何学における公理」であると思っておけばよい.}
  .
  すなわち,『原論』においては
  「公理」は「自明の事実」として扱われていたのである.

  『原論』は紀元前に書かれた幾何学の教科書である.
  しかし,後世の人々によって加筆されたり翻訳されたりなどにより,
  長い間幾何学の標準的な教科書として使われ続けてた.
  それほどまでの完成度の教科書が紀元前の時点ですでに出現していたことは
  驚くべきことであろう.

  『原論』の冒頭で定義されている用語は23個あるが,そのうちのいくつかを見てみよう.
  \begin{itemize}
    \item 点とは部分を持たないものである.
    \item 線とは幅のない長さである.
    \item 線の端は点である.
    \item 直線とはその上にある点について一様に横たわる線である.
    \item 平面とはその上にある線について一様に横たわる面である.
  \end{itemize}
  この「定義」についてはのちにもう少し深く考えることにして,
  『原論』における公準と公理も見ておこう.
  『原論』における公準は以下の通りである.
  \begin{enumerate}[公準1.]
    \item 任意の点とこれと異なるほかの任意の点とを結ぶ直線を引くことができる.
    \item 任意の線分はこれを両方にいくらでも延長することができる.
    \item 任意の点を中心とする任意の半径の円を描くことができる.
    \item 直角はすべて互いに等しい.
    \item 2直線が1直線と交わるとき,その同じ側にできる内角の和が
      2直角よりも小さいならば,2直線はその側に延長すると必ず交わる.
  \end{enumerate}
 そして,『原論』において述べられている公理は以下の通りである.
  \begin{enumerate}[公理1.]
    \item 同一のものに等しいものはまた互いに等しい.
    \item 等しいものに等しいものを加えれば,その結果もまた等しい.
    \item 等しいものから等しいものを引けば,その結果もまた等しい.
    \item 互いに重なり合うものは互いに等しい.
    \item 全体は部分より大きい.
  \end{enumerate}


  先に述べた公準と公理を眺めていると,
  明らかに第5公準だけが複雑で長ったらしい主張であることに気づく.
  この第5公準はほかの公準から導けるのではないかという疑問が出現し,
  多くの数学者がその証明に取り組んだ.
  しかし,いくら探しても第5公準がほかの公準から証明されることはなかった.
  
 \paragraph{非Euclid幾何学の登場}
  この問題に解決の兆しが見られたのは19世紀に入ってからのことである.
  \index[nidx]{Gauss@Gauss(ガウス)}
  Gaussや
  \index[nidx]{Lobachevski@Lobachevski(ロバチェフスキー)}
  Lobachevski,
  \index[nidx]{Bolyai@Bolyai(ボヤイ)}
  Bolyaiらの手により,第1公準から第4公準を仮定し,第5公準を否定する幾何学が
  提案された
  \footnote{Gaussは,宗教的な論争に巻き込まれることを恐れたのか公表はしていない.
  しかし,このような幾何学が存在するということは確信していたようである.}
  .
  このことから,第1公準から第5公準までを
  すべて仮定するような幾何学は
  \index[widx]{Euclidきかがく@Euclid幾何学}Euclid幾何学と呼ばれるようになり,
  第5公準を否定するような幾何学は
  \index[widx]{Euclidきかがく@Euclid幾何学!ひEuclidきかがく@非---}非Euclid幾何学
  と呼ばれるようになった.

  こうしてEuclid幾何学と非Euclid幾何学の両方が存在しうることが示されたわけであるが,
  Euclid幾何学や非Euclid幾何学の公理系
  \footnote{ここではとりあえず「公理の集まり」と考えておくとよい.
  また,ここでの「公理」は『原論』でいうところの「公準」である.}
  そのものが
  矛盾を抱えていないということが示されたわけではないということに注意しなくてはならない.
  矛盾が生じるような公理を仮定しても
  そこから構築される理論に(通常の意味では)価値はない.
  また,一見仮定した公理系に矛盾がないように見えても,
  そこから矛盾した結果が導かれることはないという保証はない.
  公理系の無矛盾性は決して自明なことではないのである.

  Euclid幾何学や非Euclid幾何学の公理系の無矛盾性に関しては,
  \index[nidx]{Klein@Klein(クライン)}
  Kleinや
  \index[nidx]{Beltrami@Beltrami(ベルトラミ)}
  Beltramiらの手により不完全ながらも解決されることとなった.
  その方法は「Euclid幾何学の公理系で許された手法のみを利用して,
  非Euclid幾何学の公理系を満たす対象を実際に構成してみせる」
  というものである.
  ある公理系を満たす対象のことをその公理系の
  \index[widx]{こうり@公理 \, axiom!もでる@---系のモデル \, model} 
  \emph{モデル}(model)と呼ぶ.
  すなわち,KleinやBeltramiらが行ったのは
  Euclid幾何学上に非Euclid幾何学のモデルを構築したということである.
  非Euclid幾何学のモデルとして有名なのは
  \index[nidx]{Riemann@Riemann(リーマン)}
  Riemannによって構築された球面幾何学であろう.
  それらの非Euclid幾何学の詳細は省くが,
  非Euclid幾何学の公理系を満たす対象が具体的に構成されたからには,
  非Euclid幾何学の公理系の無矛盾性を認めざるを得ない.
  「正しい」手続きのもとで構成された対象が
  満たす性質に矛盾が存在するわけがないからである.
  しかし,非Euclid幾何学のモデルはEuclid幾何学のもとで構築されたわけであるから,
  実際にいえるのは「Euclid幾何学の公理系が無矛盾であれば
  非Euclid幾何学の公理系も無矛盾である」ということである.

  非Euclid幾何学の登場により,「そもそも幾何学とは何なのか?」という疑問が出てくる.
  この問に対して1つの答えを与えたのがKleinである.
  Kleinは「空間
  \footnote{空間に関しては\ref{sec:structure}で解説する.変換群に関しては代数学の本を参照せよ.}
  $S$と変換群$G$が与えられたとき,
  $S$の部分集合,すなわち$S$上の図形に関する種々の性質のうち,
  $G$に属するすべての変換によって不変に保たれるものを研究することが
  $G$に従属する$S$の幾何学である」
  とした.
  この主張はKleinがエルランゲン大学の教授職に就くときに示されたものであり,
  \index[widx]{えるらんげんぷろぐらむ@エルランゲン・プログラム}エルランゲン・プログラム
  と呼ばれている.
  エルランゲン・プログラムの登場により,
  当時知られていたさまざまな幾何学を統一的な視点で
  考えることができるようになったのである.

 \paragraph{公理論と形式主義}
  さて,Euclid幾何学の公理系が無矛盾であるのならば
  非Euclid幾何学の公理系も無矛盾であるのであった.
  ではEuclid幾何学の公理系は無矛盾であるのか? と考えるのは当然である.
  この問に対しては,そもそもEuclid幾何学の公理系に
  論理的な不備があったことに注目しなくてはならない.
  
  \index[nidx]{Hilbert@Hilbert(ヒルベルト)}
  Hilbertは1899年に著書『幾何学基礎論』において,
  Euclid幾何学の論理的不備を正した公理系を提出した
  \footnote{現代の視点で述べるのであれば,Hilbertが提案したEuclid幾何学の公理系は
  少々扱いづらく,
  \index[nidx]{Tarski@Tarski(タルスキ)}
  Tarskiという人物が考案した公理系のほうが便利である.
  また,Tarskiによる公理系を改良する研究は今も続いているようである.}
  .
  『幾何学基礎論』においては,「点」や「直線」などの用語が
  使われているにも関わらず,その定義は書かれていない.
  これは『原論』とは大きく異なる点である.
  また,公理中で「...が---の間にある」などといった言葉が使われているが,
  これにも定義は書かれていない.
  これは,Hilbertが定義する必要のない自明な概念として用いたからではない.
  これらの用語はそもそも定義なしに用いる用語であるとしたのである.
  『原論』では,「点とは部分を持たないものである」だとか「線とは幅のない長さである」
  などといった定義がなされているが,この「部分をもたない」や「幅のない長さ」
  とは一体なんだろうかということを考えてみると,
  どういうように定義をしても結局は循環論法になってしまうことに気づく.
  そこで,Hilbertはこれらの用語を定義することなしに公理を説明した.
  このように,公理を説明するために定義せずに用いる用語を
  \index[widx]{むていぎじゅつご@無定義術語 \, undefined term}
  \emph{無定義術語}(undefined term)という.
  『幾何学基礎論』においては,
  「点」や「直線」,「平面」といった用語が無定義術語である.
  定義されないのだから,これらの用語を別のものに置き換えても理論に何ら支障はない.
  すなわち,これらの用語を「テーブル」,「椅子」,「ビールジョッキ」
  といった言葉に置き換えても理論としてはまったく同じであるとしたのである.
  この頃から,
  \index[widx]{こうり@公理 \, axiom}
  \emph{公理}(axiom)とは,
  理論の出発点となる単なる仮定であり,
  それが自明であるなどとは考えないようになった.
  この考え方に基づいて組み立てられた理論は
  \index[widx]{こうり@公理 \, axiom!こうりろん@---論 \, axiomatics}
  \emph{公理論}(axiomatics)的であるという
  \footnote{よく
    \index[widx]{こうり@公理 \, axiom!こうりしゅぎ@---主義 \, axiomatism}
    \emph{公理主義}(axiomatism)
    という言葉が使われることがあるが,これは日本だけでしか使われない.
    しかし,Hilbertの思想を表現するには便利なのでしばしば使われる.
    もちろんaxiomatismも和製英語である.}
  .しかしHilbertはさらに,公理からの「演繹的推論」にも手を出した.
  Hilbertは我々が当たり前のように使っている
  演繹的推論も定式化して議論できるのではないかと考えたのである.
  Hilbertは,公理は「意味」を持たない単なる記号列であり,
  そこから定められたルールに基づく記号操作により,
  あらゆる命題が導かれるとした.
  そして,この「記号操作」こそが
  \index[widx]{しょうめい@証明 \, proof}
  \emph{証明}(proof)であるとしたのである.
  Hilbertのこの考え方は
  \index[widx]{けいしきしゅぎ@形式主義 \, formalism}
  \emph{形式主義}(formalism)と呼ばれている.
  この資料ではHilbertが考案した演繹体系は取り扱わないが,
  代わりに第\ref{chp:sequent}章において,
  \index[nidx]{Gentzen@Gentzen(ゲンツェン)}
  Gentzenが考案した自然演繹という演繹体系を取り扱う.
  自然演繹は形式主義に基づく演繹体系のひとつである.

  理論の出発点として仮定した公理系には矛盾があってはならない.
  また,仮定した公理系にほかの公理から導かれる結果が混ざっているのは美しくない.
  すなわち,公理系は全体として矛盾を含まないという
  \index[widx]{むむじゅんせい@無矛盾性 \, consistency}
  \emph{無矛盾性}(consistency)は絶対に必要であり,
  さらに,ほかの公理から導かれる結果が混ざっていないという
  \index[widx]{どくりつせい@独立性 \, independence}
  \emph{独立性}(independence)があるのが望ましい.
  そして,公理は理論の出発点であるが,
  研究としては公理にたどり着くことはゴールであるというべきである.
  また,公理の選び方は議論の進め方によって
  いくらでも変わりうるということにも注意すべきである.

  さて,Hilbertの功績により,
  Euclid幾何学の公理系の無矛盾性が本格的に議論できるようになったわけであるが,
  その問題はすでに解決していたといってもよい.
  \index[nidx]{Descartes@Descartes(デカルト)}
  Descartesが創始した解析幾何学がEuclid幾何学のモデルになっていると考えられるからである.
  解析幾何学は実数論に基づいて構築されているから,
  このことからわかるのは「実数論の公理系が無矛盾であるならば,
  Euclid幾何学の公理系も無矛盾である」ということになる.
  
  公理論的な数学理論の作り方をまとめておこう.
  まず,基礎となる用語や概念を無定義術語として用意する.
  そして,理論の基礎となる命題を公理に据える.
  公理を仮定して論理や集合を用いて証明された命題を
  \index[widx]{ていり@定理 \, theorem}
  \emph{定理}(theorem)と呼ぶ
  \footnote{通常は定理の中でも特に重要なもののみを「定理」として挙げる場合が多い.}
  .もちろん,すでに得られた定理を利用することで新たな定理を得ることも許される.
  また,定理の中でもある定理から即座に得られるものはその定理の
  \index[widx]{けい@系 \, corollary}
  \emph{系}(corollary)と呼び,
  その定理自体にはさほど興味がなく,
  そこから導かれる結果のほうに関心があるとき,
  もととなるその定理のことを
  \index[widx]{ほだい@補題 \, lemma}
  \emph{補題}(lemma)と呼ぶことがある.
  定理,系,補題には明確な区別があるわけではなく,
  どの用語が使われるかは理論を構築する人の意図による.
  また,よく使う対象や関係を一言で表したいときには,
  それらの対象や関係に名前をつけて活用すればよい.
  名前のついた対象や関係のことを
  \index[widx]{ていぎ@定義 \, definition}
  \emph{定義}(definition)と呼ぶ.

  さて,古典的な数学観においては,現実世界における数や図形の性質を
  明らかに正しいと思われる公理に基づき,そこから正しい推論によって
  定理を証明し,その得られた定理は当然正しいとしていた.
  しかし,公理論的数学観においては,
  現実世界とはまったく無関係な無定義術語によって表現された
  公理を単なる仮定とし,そこから論理や集合を用いて
  定理を証明していくのだが,
  そもそも無定義術語に意味が存在しないため,
  公理や定理が正しいかどうかなど考えること自体が無意味である.
  もっとも,これは単なる建前上の話であり,
  実際は無定義術語に意味が存在しないとしているのは
  みずからの先入観を排除するためであり,
  現実世界とは別に存在している「数学的実在の世界」
  の様子を調べていて,その表現手段として言葉や記号を用いているのだと考えるべきである.
  「意味」とは無関係に議論しているからこそ,
  数学理論に適当な意味づけをしてやることにより,
  その手法や結果が自然科学に限らないありとあらゆる学問分野に応用できるのである.






 \section{定義の手法とそのwell-defined性}
 \label{sec:welldef}
  数学に限らず,ありとあらゆる学問分野においてそこで用いられる言葉の定義がなされ,
  それに基づいた議論が行われている.
  数学においても同じだが,少し特殊な方法で定義がなされることもある.
  後半では定義をする上で保証しなくてはならない
  well-defined性についても解説する.

 \paragraph{定義の手法}
  数学において何かを定義するとき,
  もっともよく使われる手法が「すでに定義された対象や関係,
  もしくは無定義術語の組み合わせでできる
  対象や関係に名前をつける」というものである.
  \begin{ex} \label{ex:defhutuu}
    「各辺の長さがすべて等しい三角形を正三角形という」という定義は,
    「辺」,「長さ」,「三角形」という対象と「等しい」という関係を組み合わせてできる
    「各辺の長さがすべて等しい三角形」という対象に対して
    「正三角形」という名前をつけている.
    また,「整数$a,b$に対し,$a$は$b$で割り切れる,もしくは$b$は$a$を割り切るとは,
    $a=bc$となる整数$c$が存在することをいい,このことを
    $b \mid a$と表す」という定義は,
    整数$a,b$に対する「$a=bc$となる整数$c$が存在する」という関係に
    「$a$は$b$で割り切れる」と「$b$は$a$を割り切る」という2つの名前と
    「$b \mid a$」という記号列を与えている.
  \end{ex}

  定義を書くときには「...を---という」か「---であるとは...であることをいう」
  のどちらかの形式で書くことが多い.

  対象を定義するとき,「${:=}$」や「$\overset{\mathrm{def}}{=}$」あるいは
  「$\equiv$」などの記号を用いて
  \begin{align*}
    f(x) := {x^2 + 1} , \quad 
    f(x) \overset{\mathrm{def}}{=} x^2 + 1 , \quad
    f(x) \equiv x^2+1
  \end{align*}
  などと表すこともある.これらはすべて
  「$f(x)=x^2+1$と定める」という意味である.
  また,関係を定義をするとき,記号「$\Longleftrightarrow$」を用いて
  「整数$n$に対し,
  \begin{align*}
    n \text{が偶数である} \Longleftrightarrow n \text{が2で割り切れる}
  \end{align*}
  と定める」と書かれることもある.
  最後の「と定める」は重要である.
  これがないと定義しているのか事実を述べているのかがわからないからである.
  このように定義する場合,
  \[
    \text{(新たに定義する関係)} \Longleftrightarrow \text{(既知の関係)}
  \]
  という形式で書かれることが多い.

  定義の手法でよく使われるものがもうひとつある.
  それは「定義する対象や関係を構成する方法を提示する」というものである.
  このような定義を %
  \index[widx]{ていぎ@定義 \, definition!さいきてきていぎ@再帰的---|see{帰納的定義}}
  \emph{再帰的定義} もしくは
  \index[widx]{ていぎ@定義 \, definition!きのうてきていぎ@帰納的--- \, recursive ---}
  \emph{帰納的定義}(recursive definition)という.
  \begin{ex} \label{ex:defkinou}
    命題論理式を以下に述べるように帰納的に定義する:
    \begin{enumerate}[(1) ]
      \item 命題変数は命題論理式である.
      \item $\curlyvee$と$\curlywedge$は命題論理式である.
      \item $\varphi$が命題論理式であるとき,$\lnot \varphi$は命題論理式である.
      \item $\varphi , \psi$が命題論理式であるとき,
        $\varphi \land \psi , \varphi \lor \psi , \varphi \to \psi , 
        \varphi \rightleftarrows \psi$
        はすべて命題論理式である.
      \item 以上の規則を有限回適用して得られるもののみが命題論理式である.
    \end{enumerate}
  \end{ex}

  例\ref{ex:defkinou}では,「命題論理式」という用語を
  \begin{enumerate}[1. ]
    \item 出発点となる命題論理式を用意する.
    \item すでに得られた命題論理式から別の命題論理式をつくる方法を述べる.
    \item 以上の操作を有限回適用して得られるもののみが命題論理式であると宣言する.
  \end{enumerate}
  という手順に従って定義している.
  この3.のステップはどこまでが命題論理式なのかを指定する重要な
  ステップであるが,明らかであるとして省略されることも多い.
  \begin{ex} \label{ex:zenkasiki}
    数列$\{ a_n \}$を漸化式
    \begin{align*}
      a_1 & = 1 \\
      a_{n+1} & = \sqrt{ a_n + 6} \quad ( n \in \mathbb{N} ) 
    \end{align*}
    で帰納的に定義する.
  \end{ex}

  例\ref{ex:zenkasiki}では,$a_1$がまず与えられ,そこから$a_2 , a_3 , \ldots$
  が順にすべて得られる等式が提示されている.
  この等式により,数列$\{ a_n \}$の任意の項を求めることができる.

  例\ref{ex:defkinou}と例\ref{ex:zenkasiki}からもわかるように,
  帰納的定義においては定義の中に定義したい概念が登場しているという特徴がある.

 \paragraph{well-defined性}
  定義というのは対象や関係に関する単なる約束事であるから
  (一般によく使われるものを除けば)
  何にどんな名前をつけるかは個々人の自由である.
  しかし,定義であればどんなものでも許されるわけではない.
  \begin{ex} \label{ex:welldef1}
    「実数$\alpha$を
    $\displaystyle \alpha = \lim_{n \to \infty} n$
    と定める」という定義は不適切である.
    なぜならば,右辺は明らかに実数ではなく,
    この$\alpha$は「実数」とは呼べないからである.
    「実数$x$に対し,$y^2= x$を満たす実数$y$を対応させる関数を$f(x)$とする」
    という定義も不適切である.
    $x$が負であれば$y^2=x$を満たす実数$y$は存在せず,
    $x$が正であれば$y^2=x$を満たす実数$y$は2つ存在する.
    よって,この対応関係は「関数」と呼ぶことはできない.
  \end{ex}

  このように,すでに定義されたものを使って新しく定義をする場合,
  その定義がそれまでの議論と矛盾しないことを保証しなくてはならない.
  このことが保証されている場合,その定義は
  \index[widx]{well-defined@well-defined}
  \textbf{well-defined}であるという.
  また,定義がwell-definedでない場合,その定義は
  \index[widx]{ill-defined@ill-defined}
  \textbf{ill-defined}であるという.
  例\ref{ex:welldef1}で挙げた定義はともにill-definedである.

  \begin{ex}
    「実数$\gamma$を
    \begin{align*}
      \gamma = \lim_{n \to \infty} \left( \sum_{k=1}^{n} \frac{1}{k} - \log n \right)  
    \end{align*}
    と定める」という定義はwell-definedである.
    なぜならば,右辺の極限が確かに有限の実数であることが示せるからである.
  \end{ex}
  well-defined性について身近なのは,代数学において同値類に演算を導入する場面であろう.
  \begin{ex}
    2以上の自然数$n$に対し,$n$を法とする剰余類
    全体の集合$\mathbb{Z} / n \mathbb{Z}$上に加法$+$を
    \begin{align*}
      [a] + [b] = [ a+b ] \quad \left( [a] , [b] \in \mathbb{Z} / n \mathbb{Z} \right)
    \end{align*}
    で定義する.このとき,この加法$+$はwell-definedである.
    なぜならば,よく知られているように,
    与えられた$[a] , [b] \in \mathbb{Z} / n \mathbb{Z}$に対して
    $[a+ b]$はその代表元のとり方によらず一意に定まることが示せるため,
    この加法は確かに「集合$\mathbb{Z}/n \mathbb{Z}$上に定義された加法」
    といえるからである.もしも結果が剰余類だけでなくその代表元に依存するならば,
    これは$\mathbb{Z} / n \mathbb{Z}$の2つの元を定めただけではその結果が確定しないことになる.
    それでは「$\mathbb{Z}/ n \mathbb{Z}$上に加法を定めた」とはいえない.
  \end{ex}

  このように,数学において何かを定義する場合,その
  well-defined性は必ず確かめなくてはならない.
  しかし,そのことは明らかであるか,
  明らかでなくとも容易であるとして省略されることも多い.


\section{数学的構造}
\label{sec:structure}
  
  19世紀以降,数学理論の考察の対象は,
  数や図形といった具体的なものから集合という根源的な
  ものへと移り変わり,大きく抽象化されていくこととなった.
  その原動力となったのが,
  CantorとDedekindによって創始された集合論と,
  \index[nidx]{Galois@Galois(ガロア)}
  Galois以来大きな発展を遂げた群論を中心とする代数学である.
  
  この数学の抽象化という流れと相性が良かったためか,
  Hilbertの提唱した形式主義という考え方が
  幾何学のみならず数学全体に広まることとなった.
  ところが,Hilbertの思い描いた「公理」と
  我々が現在思い描く「公理」には微妙に差異がある.
  典型的な例として,以下に挙げる群の公理が挙げられる.
  \begin{axiom}[群の公理] \label{axiom:group}
    空でない集合$G$と$G$の元$e,$そして$G$上の
    二項演算$\ast : G \times G \longrightarrow G$と単項演算${} ^{-1} : G \to G$
    が以下の条件を満たすとき,
    対$(G,e, \ast , {} ^{-1})$を群と呼び,$G$は演算$\ast$に関して$e$を単位元とする
    群をなすという:
    \begin{enumerate}[(1) ]
      \item $\forall x, y, z \in G (x \ast ( y \ast z)) ,$
      \item $ \forall x \in G (x \ast e = e \ast x = x),$
      \item $\forall x \in G (x \ast x^{-1} = x^{-1} \ast x = e).$
    \end{enumerate}
  \end{axiom}
  Hilbertが思い描いた「公理」というのは,
  あくまで確定した1つの対象を定式化するための命題である.
  しかし,群の公理系のモデルのなかで明らかに「異なる」と思われるものはいくつも発見できる.

  \begin{ex} \label{ex:groupmodel}
    (空でない)集合$X$に対し,$X$から自身への全単射全体の集合は,
    写像の合成に関して恒等写像を単位元とする群をなす.
    また,整数全体の集合$\mathbb{Z}$は,その加法に関して
    $0$を単位元とする群をなす.
    これらは「同じ」群であるが,両者は明らかに「異なる」と思われる.
    実際,前者は演算が可換ではないが,後者は演算が可換である.
  \end{ex}

  群の公理とは少し様子が異なる公理系を挙げておこう.

  \begin{ex} \label{ex:hilbertaxiom}
    実数論の公理系のモデルはどの2つも「本質的に同じ」である.
    すなわち,実数論の公理系のモデルは本質的にただひとつである.
  \end{ex}

  Hilbertは,「よい」公理系に対し,無矛盾性や独立性だけではなく,
  この「その公理系のモデルが本質的にただひとつである」という性質も要求していた.
  なぜそのような性質を要求したかは省略するが,
  Hilbertにとっては,群の公理やベクトル空間の公理といった
  同じ公理系のモデルで「異なる」モデルが存在するような公理系は
  「よい」公理系ではなかったことになる.

  公理系のモデルが「本質的に同じである」とき,そのモデルは
  \index[widx]{どうけい@同型 \, isomorphic}
  \emph{同型}(ismorphic)であるという.
  さらに,公理系のモデルがどの2つも同型である,すなわち
  その公理系のモデルが本質的にただひとつであるとき,
  その公理系は
  \index[widx]{はんちゅうせい@範疇性 \, categoricity}
  \emph{\ruby{範疇性}{はんちゅうせい}}(categoricity)
  をもつという.
  「本質的に同じである」というのがどういうことなのかということを
  述べるのはやや面倒なので省略するが,
  公理系に範疇性を要求していたHirbertにとっては,
  「公理系を考察する」ということは「ある特定の具体的な対象を考察する」
  ことと同じだったのである.

  ところが,Hilbertのこのような態度は21世紀を生きる我々とは相容れないものがある.
  我々は公理系を考察するとき,「この公理系を満たす対象全体の共通点を探っていく」
  という態度をとっているはずである.
  これは,暗に「公理系を満たす対象は複数存在する」
  ということを認めているということを意味する.
  すなわち,公理を,ある特定の具体的な対象を定式化するためではなく,
  無数の対象に共通する性質を抽出して定式化するために用いているのである.
  
  このような「無数の対象に共通する性質を探っていく」という態度で
  数学理論を最初に構築したのは誰か,という疑問に答えるのは実に難しい.
  しかし,「無数の対象に共通する性質を探っていく」ことそのものに着目し,
  それを最初に定式化したのはおそらく
  \index[nidx]{Bourbaki@Bourbaki(ブルバキ)}
  Bourbakiであろう.

  フランスの数学者集団
  \footnote{Bourbakiは多数の若手数学者からなる集団であったが,
    当初はさも個人であるかのように活動していた.説明のしやすさを考え,
    ここでもBourbakiをさも個人であるかのように扱う.}
  Bourbakiは,19世紀のCauchyの時代以降標準的に用いられていた解析学の教科書に
  不満を持っていた.
  そこで,現代的な解析学の教科書を執筆しようということになったのだが,
  その作業があまりに膨大であったため,最終的には現代数学を
  厳密かつ公理論的に構築し直そうということになった.
  そうして1939年に『数学原論』という本の第1巻が出版されることとなる.

  「原論」とあるように,『数学原論』はただひたすらに一般から特殊へという
  流れで議論が進んでいく.
  しかし,Euclidの『原論』の時代とは当時得られていた数学の成果の量が段違いであったため,
  その量も『原論』とは比べ物にならない.
  日本語に翻訳されているものだけでも30巻はゆうに超え,
  今もなお続刊が執筆され続けている.
  Euclidの『原論』と同じように,『数学原論』で語られる事実の多くは
  Bourbakiが発見したものではなく,当時すでに得られていたものである.
  『数学原論』において画期的なのは,その考察対象を
  集合に対してある種の「性質」を付与するという方法で構成していったことである.

  集合に対してある種の「性質」を付与したものを
  \index[widx]{くうかん@空間 \, space}
  \emph{空間}(space)と呼び
  \footnote{この言い方だと群や環といった我々が素朴な意味で「空間」と呼んでいるものとは
    明らかに異なったものも「空間」ということになってしまうが,明示的に「空間」と
    呼ばれるのはベクトル空間や距離空間などの我々の幾何学的直感が
    通用するものがほとんどである.}
  ,
  付与するその「性質」のことを
  \index[widx]{すうがくてきこうぞう@数学的構造 \, mathmatical structure}
  \emph{数学的構造}(mathmatical structure)
  ,あるいは単に %
  \index[widx]{こうぞう@構造 \, structure|see{数学的構造}}
  \emph{構造}(structure)
  と呼ぶ.
  空間を考える文脈では,構造を与える前の集合のことを
  その空間の
  \index[widx]{だいしゅうごう@台集合 \, underlying set}
  \emph{台集合}(underlying set)という.
  そして,この数学的構造が満たすべき条件こそが
  現在でいうところの「公理」なのである.
  かなり曖昧な言い方であるが,明確な形で定義を述べるのは
  かなり面倒であるため省略させてもらうことにする.

  Bourbakiが導入した数学的構造の中で最も基本となるのは,
  代数構造・順序構造・位相構造の3つである.
  これらの構造を複数もっているとみなせる集合は少なくなく,
  たとえば実数全体の集合$\mathbb{R}$は上記3つの構造をすべてもっていると解釈できる.
  
  Bourbakiは「集合の上に構造を付与する」という思想の上で
  数学を構築していった.すなわち,構造を付与する前の集合は,
  何の意味ももたない単なる「モノ」の集まりということである.
  従って,集合を特徴づけるのは「モノ」の数のみということとなり,
  それさえ同じなら「集合としては」同じということになる.

  \begin{ex} \label{ex:setisomorphic}
    自然数全体の集合$\mathbb{N}$と有理数全体の集合$\mathbb{Q}$
    の間には全単射が存在する.
    集合の濃度をその集合の個数と解釈するのならば,
    $\mathbb{N}$と$\mathbb{Q}$は「集合としては」まったく「同じ」ということになる.
  \end{ex}

  「$\mathbb{N}$と$\mathbb{Q}$がまったく同じ」と言われれば,
  ほとんどの人が違和感を抱くに違いない.
  しかしBourbaki流にいえば,その「違い」は$\mathbb{N}$と$\mathbb{Q}$を単なる集合
  とだけ考えたのでは決して発見できず,そこに付与された「構造」を見ることにより
  初めて発見できるということになる.
  実際,$\mathbb{N}$と$\mathbb{Q}$は(自然に持ち合わせているとみなせる)
  代数構造,順序構造がまったく異なる.
  代数構造が異なる根拠としては$\mathbb{Q}$は体であるが$\mathbb{N}$は
  体でないこと,順序構造が異なる根拠としては
  $\mathbb{N}$には最小元が存在するが$\mathbb{Q}$には最小元が存在しないこと
  などが挙げられる.

  「集合として」同じであっても,「構造」が違えばそれらの対象は
  異なるとみなされる.一方,
  「集合として」同じであり,かつその「構造」も同じであれば,
  それらの対象はたとえどのような形で記述されていたとしても「同じ」と
  みなされる.

  \begin{ex} \label{Rseidoukei}
    実数全体の集合$\mathbb{R}$と正の実数全体の集合$\mathbb{R}^+$は,
    $\mathbb{R}$についてはその加法,$\mathbb{R}^+$については
    その乗法の代数構造のみを考えると,
    これら2つは集合としても「同じ」で,代数構造も「同じ」である.
    実際,$\mathbb{R}$上で$\alpha , \beta$に対して$\alpha + \beta$
    という和を考えることは,$\mathbb{R}^+$上においては
    $e^{\alpha} \cdot e^{\beta}$という積を考えることに相当する.
  \end{ex}

  例\ref{Rseidoukei}が意味することは,加法群$( \mathbb{R} , 0, +,-)$と
  乗法群$( \mathbb{R}^+, 1, \cdot,{}^{-1}) $が「本質的に同じ」
  である,すなわちこの2つの群は群として同型であるということである.
  このことは通常$(\mathbb{R},0,+, -) \cong (\mathbb{R}^+ ,1,\cdot , {}^{-1})$
  と表記される.「群」という観点で考える限り,
  これら2つの対象は「同じ」とみなされる.
  これは,群論がこれら2つの対象に共通する性質のみを研究対象とすることを意味する.
  幾何学においては,エルランゲン・プログラムがちょうどそのような視点に沿って
  幾何学を構築すべきだと主張するものである.
  互いに移り合う変換群が存在するような2つの図形を同型と考えるのである.

  2つの対象が同型であることは,全単射であって,その写像と逆写像の両方が
  構造を保つものが存在することとして定式化される.
  同種の構造をもった2つの集合$A,B$において,
  全単射とは限らないが構造を保つような写像$f:A \longrightarrow B$が存在する場合,
  その写像$f$のことを
  \index[widx]{しゃぞう@写像 \, mapping!じゅんどうけいしゃぞう@準同型--- \, homomorphism}
  \emph{準同型写像}(homomorphism)という.
  準同型写像$f$が特に単射である場合,$A$と$f(A)$が同型である場合がある.
  このとき,$f$を
  \index[widx]{うめこみ@埋め込み \, embedding}
  \emph{埋め込み}(embedding)という.
  埋め込みが存在する場合,$A$と$f(A)$を同一視して,$A$が$B$の部分集合であるかのように
  扱うことができる.

  もちろん,群として同型であっても,他の構造に目を向ければ同型でないということはある.

  \begin{ex} \label{ex:Rseinotdoukei}
    $\mathbb{R}$には整列順序と呼ばれる順序構造$\leq '$を入れ,
    $\mathbb{R}^+$には通常の大小関係と同じ順序構造$\leq$を入れる.
    このとき,これら2つの順序集合は順序同型でない.
    実際,順序集合$(\mathbb{R}, \leq ')$は最小元をもつが,
    順序集合$( \mathbb{R}^+ ,\leq) $は最小元をもたない.
  \end{ex}

  このように,同型でない複数の対象が同じ公理系のモデルとして議論されているのが
  Bourbaki流の公理論的数学理論の最も大きな特徴と考えることができる.
  そのため,典型的な例とは明らかにかけ離れた対象も同じ土台に乗せられて
  議論されることがある.
  たとえば,$n$次元数ベクトル空間$\mathbb{R}^n$とある閉区間$[a,b]$上で定義された
  関数全体の集合$\mathbb{R} ^{[a,b]}$をベクトル空間とみなしたものが
  同じ(同型という意味ではない)
  ベクトル空間とみなされ,一緒くたにされて議論されていることには
  違和感を抱くこともあるだろう.
  もちろんこれら2つのベクトル空間は同型ではないため,
  「この2つは違う対象だろう」という直感は健全なものである.

  Bourbakiのこの「集合というまっさらな土台に構造を与え,
  それにより生まれる性質を研究する」という思想はすぐに数学全体に波及した.
  しかし,波及した学問分野は数学だけにとどまらなかった.
  フランスの文化人類学者
  \index[nidx]{Strauss@Strauss(ストロース)}
  Straussがオーストラリアの原住民の婚姻システムの「構造」を
  群論を用いて説明してみせたのである.
  このことを皮切りに,Bourbakiの思想は
  \index[widx]{こうぞう@構造 \, structure!しゅぎ@---主義 \, structuralism}
  \emph{構造主義}(structuralism)という名前で世界中に広まり,
  数学に限らない数多くの学問分野に取り入れられることとなった.

  現在では,数学的構造とその間の関係を研究する圏論や,
  数理論理学の手法を用いて数学的構造を研究するモデル理論といった
  数学的構造そのものが研究対象となっている理論も生み出され,
  そして発展している.
  残念ながら『数学原論』はこれらをカバーしてはいないが,
  構造主義という思想がこれらの理論に影響を与えていることは
  もはや疑う余地はないように思える.
  





  


  

  

\section{よく使う数学界の方言}
\label{sec:hougen}

 % 公理論
 \chapter{記号論理超入門}
\label{chp:sequent}
%
%%%%%%%===========イントロダクション============%%%%%%%%%%%

数学理論は論理の上に成り立っている.
このことは疑いようのない事実であろう.
基礎となる公理を出発点とし,
演繹的推論を適用して証明を積み重ねていく.
これが最も一般的な数学的議論のフォーマットである.

しかし,この「論理」や「推論」に関して
重要な成果が挙げられ始めたのは案外最近で,
20世紀あたりからである.有名どころでいえば,
\index[nidx]{Godel@G\"{o}del(ゲーデル)}G\"{o}delの不完全性定理が挙げられる.
もちろん,他にも重要で興味深い話題はたくさんある.

この資料ではそのあたりの面白くて興味深い話題は取り扱わないが,
そこに至るまでの道筋のほんの小さな一歩を踏みしめてみよう.

なお,この資料では
syntaxとsemanticsという2つの立場の区別や,
基本的な用語の厳密な定義はあまり重視していない.
そのため,数理論理学の視点から見るとかなり不満のある
議論が多く含まれていることに留意されたい.

%
%%%%%%%%%================イントロダクション終わり=====================%%%%%%%%%%%%%%%%%
%
%
%
 \section{命題論理と述語論理}
 \label{sec:ronri}
 %
 \paragraph{命題と条件}
  正しいか正しくないかが明確に定まる主張(statement)や式(expression)
  のことを\index[widx]{めいだい@命題 \, proposition}
  \emph{命題}(proposition)といい,
  ある命題について,その命題が正しいことをその命題は %
  \emph{真}(true)である,
  その命題が正しくないことをその命題は %
  \emph{偽}(false)であるという
  \footnote{「命題$A$が成り立つ」といえば,
  それは命題$A$が真であることを意味することが多い.}
  .
  
  命題は,$A,  B$や$p,  q$などといったアルファベットで表すことが多い.

  \begin{ex}
    「6は偶数である」や「1=1」は真な命題であり,
    「$3<2$」や「10は素数である」は偽な命題である.
    しかし,「7」や「偶数である」,「$x+y=0$」などといった
    ものは命題でない.
  \end{ex}
  
  最後の例「$x+y=0$」は,そのままでは命題とはならないが,
  この$x,  y$に具体的な値を代入すると命題になる.
  たとえば,$x=1,  y=2$とすると「$1+2=0$」という偽な命題になるが,
  $x=-1,  y=1$とすると「$-1+1=0$」という真な命題となる.
  このように,変数や文字を含んだ式で,
  その変数に値を代入したときに命題になるものを
  その変数に関する
  \index[widx]{じょうけん@条件 \, condition}
  \emph{条件}(condition)もしくは
  その変数に関する
  \index[widx]{めいだいかんすう@命題関数 \, propositional function|see{条件}} %
  \emph{命題関数}(propositional function)と呼ぶ.
  
  変数$x$に関する条件は,よく$F(x)$などと表される.
  2変数$x,  y$に関する条件ならば,$F(x, y)$などと表されることが多い.
  $F(x)$が「$x$は偶数である」という条件を表すのであれば,
  $F( \ )$はちょうど「...は偶数である」という部分に相当する.
  そういうわけで,変数に関する条件を
  その変数に関する
  \index[widx]{じゅつご@述語 \, prodicate|see{条件}} %
  \emph{述語}(prodicate)と呼ぶことも多い.
  $x$の条件$F(x)$について,$x$に別の変数,もしくは定数$a$を当てはめた
  ものは$F(a)$と表される.
  実数$x$に対し,
  $F(x)$が「$x>1$」を表すのだとすれば,$F(0)$は「$0>1$」を表すのである.

  \begin{ex}
    「$x$は偶数である」や「$x^2=1$」は1変数$x$に関する条件である.
    また,「$A$は正方行列である」や「$A$は正則である」
    は1つの行列$A$に関する条件である.
    さらに,「$x,  y$の少なくとも一方は正である」や「$x,  y$はともに整数である」
    は2変数$x,  y$に関する条件である.
  \end{ex}

  上の例のように,「変数」と書いたのはあくまで象徴的な意味合いであって,
  本当の意味での「数」でなくてもかまわない.
  行列に関する条件や,関数に関する条件なども考えられる.
  ここで言うところの「変数」とは,とりあえずいまのところは
  単なる「モノ」程度の認識でよい.
  これを強調するため,この「モノ」のことを
  \index[widx]{たいしょう@対象 \, object}
  \emph{対象}(object)やら
  \index[widx]{こう@項 \, term}
  \emph{項}(term)と呼ぶことがある.
  「対象」および「項」に関してはもう少しきちんと考えなくてはならないのだが,
  この資料ではこのあたりの用語はあまり意識せずに使うことにする.

  

  \paragraph{命題結合記号}
  ここからは,与えられた命題(もしくは条件)から
  別の命題(条件)を作り出すことを考えてみよう.
  
  命題(条件)$A$に対し,「$A$でない」という命題(条件)を
  $A$の
  \index[widx]{めいだい@命題 \, proposition!のひてい@---の否定 \, negation of ---}
  \emph{否定}(negation)といい,
  \begin{equation}
   \lnot A
    \label{eq:negation}
  \end{equation}
    と表す
  \footnote{高校の教科書では,命題$A$の否定を$\overline{A}$と表すのが一般的であるが,
  この資料では用いない.}.
  命題(条件)$A$に対し,$\lnot A$は,$A$が真である場合に偽となり,
  $A$が偽である場合に真となる.
  
  \begin{ex}
    「$1=0$」は偽な命題であり,その否定「$\lnot (1=0)$」は真な命題である.
    これを通常「$1 \neq 0$」と略記している.

    $x$を整数を表す変数として,
    $x$に関する条件「$x$は偶数である」
    の否定「$\lnot ( x \text{は偶数である})$」は
    「$x$は奇数である」と同じ意味である.
    「$\lnot ( x \text{は偶数である})$」は$x$が奇数である場合に真となり,
    $x$が偶数である場合に偽となる.
  \end{ex}

  2つの命題(条件)$A,  B$に対し,「$A$かつ$B$」という命題(条件)を
  $A$と$B$の
  \index[widx]{めいだい@命題 \, proposition!のれんげん@---の連言} %
  \emph{連言}(conjunction)といい,
  \begin{align}
   A \land B 
    \label{eq:conjunction}
  \end{align}
  と表す.
  命題(条件)$A,  B$に対し,
  $A \land B$は$A,  B$がともに真であるとき,
  またそのときにのみ真となり,
  他の場合は偽となる.

  \begin{ex}
    2つの命題「$1=1$」と「$2=2$」はともに真である.
    従って,命題「$1=1 \land 2=2$」は真となる.
     命題「$1=2$」は偽である.よって命題「$1=2 \land 2=2$」は偽である.

     また,「$x=1 \land y=2$」は2変数$x,  y$に関する条件である.
     これを
     \begin{align*}
       \left\{
         \begin{aligned}
          x = 1 \\
          y = 2
         \end{aligned}
       \right.
     \end{align*}
     や「$(x,y)=(1,2)$」あるいは「$ x = 1,  y=2$」などと略記することが多い.
  \end{ex}

  2つの命題(条件)$A,  B$に対し,「$A$または$B$」という命題(条件)を
  $A$と$B$の
  \index[widx]{めいだい@命題 \, proposition!のせんげん@---の選言 \, disjunction of ---}
  \emph{選言}(disjunction)といい,
  \begin{align}
    A \lor B
    \label{eq:disjunction}
  \end{align}
  と表す.
  命題(条件)$A,  B$に対し,
  $A \lor B$は$A$と$B$のどちらか一方でも真であれば真となる.
  $A \lor B$が偽になるのは$A$と$B$がともに偽である場合に限るということにする.
  これは,直感には反することである.
  たとえば,「パンまたはライスが選べます」と言われたら,
  「パン」か「ライス」のどちらか一方のみが選べると解釈するのが普通である.
  しかし\.数\.学\.に\.お\.い\.て\.は,
  「パン」と「ライス」の両方を選んでもいいのである.
  こういうふうに約束するのは,
  単にその方が数学の議論を展開しやすいからであって,
  数学独自のローカルルール(それでも数学全体におおよそ通用するが)であることに気をつけよう.
  まさか現実で「パンとライス両方で」などと頼む人はおるまい.

  \begin{ex}
    命題「$1=2 \lor 2=2$」は真である.命題「$2=2$」が真であるからである.
    また,命題「$1=2 \lor 2=3$」は偽である.
    命題「$1=2$」と「$2=3$」がともに偽であるからである.
    そして,命題「$1=1$」と「$3<4$」がともに真であるから,
    命題「$1=1 \lor 3<4$」は真となる.

    「$x=1 \lor x=3$」は1変数$x$に関する条件である.
    これを通常「$x = 1, 3$」と略記して書くことが多い.
  \end{ex}

  命題(条件)$A,  B$に対し,「$A$ならば$B$」,
  すなわち「\.も\.し$A$が真\.だ\.と\.す\.ると
  $B$も\.必\.ず真になる」という命題(条件)を
  $A$の$B$による\index[widx]{めいだい@命題 \, proposition!のがんい@---の含意 \, implication}
  \emph{含意}(implication)といい,
  \begin{align}
    A \to B
    \label{eq:implication}
  \end{align}
  と表す.
  高校の教科書では,「$\Longrightarrow$」という記号を用いて
  $A \Longrightarrow B$と表すことが多いが
  \footnote{実は,高校の教科書では命題$A,B$に対して$A \to B$という命題は議論されていない.}
  ,
  記号「$\Longrightarrow$」は違う意味で使いたいのでここでは「$\to$」を用いる.
  命題(条件)$A,  B$に対し,$A \to B$の真偽には注意が必要である.
  結論から言って,命題(条件)$A,  B$に対し,$A \to B$は
  $A$が真であり,かつ$B$が偽であるとき,またそのときにのみ偽となり,
  残りの場合にはすべて真となる.このことに関しては\ref{sec:sequent}で考えることにする.

  \begin{ex}
    命題「$1=2 \to 2=1$」は真な命題である.また,命題「$2=2 \to 1=4$」は偽である.
  \end{ex}

  命題(条件)$A,  B$に対し,命題(条件)$(A \to B) \land ( B \to A) $を
  $A$と$B$の\index[widx]{めいだい@命題 \, proposition!のどうち@---の同値 \, equivalence}
  \emph{同値}(equivalence)命題といい,
  \begin{align}
    A \rightleftarrows B
    \label{eq:equivalence}
  \end{align}
  と表す.
  この記号も高校の教科書では「$\Longleftrightarrow$」が使われることが多いが,
  この資料では用いない.
  命題(条件)$A,  B$に対し,$A \rightleftarrows B$は
  $A$と$B$の真偽が一致した場合にのみ真となる.
  
  以上で使った記号$\land ,  \lor ,  \to ,  \rightleftarrows$
  は,2つの命題(条件)を1つの命題(条件)として「くっつける」役割を果たしている.
  これらの記号を\index[widx]{めいだい@命題 \, proposition!
  けつごうきごう@---結合記号 \, proposional connectiove}
  \emph{命題結合記号}(propositional connective)という.
  
  $( A \to B) \land ( B \to A) $のように,
  論理記号を複数使って複雑な命題を作ることがある.
  このとき,無用な混乱を避けるため,
  どの命題結合記号がどの命題を結合しているのかを明らかにしなくてはならない.
  カッコを用いるのが普通である.
  \begin{ex}
    命題$A,  B,  C$に対し,
    $A \lor B \land C$といった書き方は許されない.
    $A \lor ( B \land C)$か$(A \lor B ) \land C$と書き表す.
    そして,これら2つの命題は一般には異なる意味を持つ.
    気になる人は$A,  B,  C$の真偽を適当に定めてみると良い.

    誤解のない範囲ではカッコは省略することが許される.
    たとえば,$A \land ( B \land C)$と$(A \land B)\land C$は
    どちらの意味に解釈されてもかまわないので,$A \land B \land C$と略記する.
    ただし,カッコの省略はあくまで「誤解のない範囲で」である.
    $A \to B \to C$などと書くのは誤解を招くので許されない.
    $( A \to B ) \to C$か$A \to ( B \to C)$と書かねばならない.
  \end{ex}
  
  命題結合記号に優先順位を設ければ,カッコの数を大幅に減らすことができる.
  そこで,命題結合記号の結合の強さを
  $\lnot$が最強,$\land,  \lor$が次,$\to ,  \rightleftarrows$が最も弱い
  と約束しよう.
  \begin{ex}
    $\lnot A \land B$は$\lnot( A \land B)$ではなく$(\lnot A ) \land B$
    と解釈する.
    また,$\lnot A \to B \lor C$は$(\lnot A \to B ) \lor C$
    などではなく$(\lnot A) \to ( B \lor C)$と解釈する.
  \end{ex}

  \paragraph{限定記号}
  $x$の条件$F(x)$に対し,「すべての$x$について$F(x)$である」という命題を
  \begin{align}
    \forall x F(x)
    \label{eq:forall}
  \end{align}
  と表す.
  記号$\forall$は\index[widx]{ぜんしょうきごう@全称記号 \, universal quantifier}
  \emph{全称記号}(universal quantifier)と呼ばれており,
  「For \underline{a}ll $x$, $F(x)$.」の「A」をひっくり返したものであると覚えておけばよい.
  $F(x)$は変数$x$に関する条件であるが,$\forall x F(x)$は
  1つの命題であることに注意してほしい.

  $x$の条件$F(x)$に対し,$\forall x F(x)$の形の命題を
  \index[widx]{めいだい@命題 \, proposition!ぜんしょう@全称--- \, universal ---}
  \emph{全称命題}(universal proposition)という.
  \begin{ex}
    命題「$\forall x ( x \geq 0 )$」は$x$が実数を表す変数である場合には偽であるが,
    $x$が自然数を表す変数である場合には真である.
  
    $y$に関する条件$「\forall x ( y \leq x^2)$」は,
    $x$が実数を表すのであれば「$ y \leq 0$」と同じ意味である.
  \end{ex}
  2番目の例について,$x$がとりうる範囲が決まれば,
  この命題の真偽は$y$によってのみ決まることに注意してほしい.

  $\forall x F(x)$の表し方には人によってさまざまで,
  「どんな$x$についても$F(x)$である」とか「任意の$x$に対して$F(x)$である」
  なども$\forall x F(x) $と同じ意味である.

  $x$の条件$F(x)$について,「$F(x)$を満たす$x$が存在する」という命題を
  \begin{align}
    \exists x F(x)
    \label{eq:exists}
  \end{align}
  と表す.
  記号$\exists$は\index[widx]{そんざいきごう@存在記号 \, existential quantifier}
  \emph{存在記号}(existential quantifier)
  と呼ばれており,「There \underline{e}xists a $x$ such that $F(x)$.」
  の「E」をひっくり返したものであると覚えるのがよいだろう.
  
  $x$の条件$F(x)$に対し,$\exists x F(x)$の形の命題を
  \index[widx]{めいだい@命題 \, proposition!そんざい@存在--- 
  \quad existential ---}
  \emph{存在命題}(existential proposition)という.
  \begin{ex}
    命題「$\exists x ( x^2 = -1)$」は$x$が実数を表す変数の場合には偽であり,
    $x$が複素数を表す変数の場合には真である.
  \end{ex}

  $\exists x F(x)$にも色々言い方がある.
  「ある$x$について$F(x)$である」だとか「$x$をうまくとれば$F(x)$とできる」
  や「$x$が存在して$F(x)$となる」などだろうか.
  このうち,「$x$が存在して$F(x)$となる」は今後よく使うだろう.
  しかし,存在命題のことを「ある$x$に対して...」と表現するのはおすすめしない.
  「存在」というニュアンスが薄れてしまうからだ.

  全称記号と存在記号を組み合わせる場合には注意が必要である.
  \begin{ex}
    $x,  y$を自然数を表す変数とすると,
    命題「$\forall x ( \exists y ( x \leq y ))$」は真である.
    任意にとった自然数$x$に対して$y=x+1$とおけば,$y$も自然数であり,
    さらに$x \leq y$が成り立つからである.

    しかし,$x,  y$を自然数を表す変数であるとして,
    全称記号と存在記号の順番を入れ替えた
    命題「$\exists y ( \forall x ( x \leq y))$」は偽である.
    どのように$y$をとろうとも「任意の$x$に対して$x \leq y$」となるようには
    できないからである.

    ただし,順番さえ守れば「$\forall x \exists y ( x \leq y) $」や
    「$\exists y \forall x ( x \leq y )$」など
    とカッコを省略してしまうのは許されるであろう.
  \end{ex}
  
  「$\exists x \forall y F(x, y)$」という形の命題を言葉で表現するとき,
  「すべての$y$に対して$F(x, y)$となる$x$が存在する」と表現することが考えられる.
  しかし,これは「``すべての$y$に対し''$F(x, y)$となる$x$が存在する」
  なのか「``すべての$y$に対し$F(x,y)$となる''$x$が存在する」なのか紛らわしい.
  今回表したいのは後者なので,
  「すべての$y$に対して$F(x,y)$となる,$x$が存在する」などと「,」
  を使って区切ってやるという手法が思いつく.
  「すべての$y$に対し,$F(x,y)$となる$x$が存在する」と書けば
  これは「$\forall y \exists x F(x,y)$」を表しているのだと解釈するのが自然だろう.
  とはいえ,このような区別は面倒だし間違えやすいので,
  「$x$が存在して,すべての$y$について$F(x,y)$」だとか
  「すべての$y$に対して$x$が存在して$F(x,y)$」などと表してしまうことにしまおう.
  日本語的には多少変かもしれないが,こっちの方が誤解は少ないと思う.

  $\forall$と$\exists$の順番は入れ替えてはいけないが,
  $\forall$同士や$\exists$同士なら話は別である.
  2変数$x,  y$に関する条件$F(x, y)$に対し,
  2つの全称命題「$\forall x \forall y F(x, y)$」と
  「$\forall y \forall x F(x, y)$」,
  および2つ存在命題「$\exists x \exists y F(x, y)$」と
  「$\exists y \exists x F(x, y)$」は同じ意味である.
  従って,これらを「$\forall x, y F(x, y)$」や
  「$\exists x, y F(x, y)$」と略記しても誤解は生じないだろう
  \footnote{全称記号同士の順序の交換はともかく,
  存在記号同士の順序の交換は直感的には納得しがたいが,
このことは\ref{sec:sequent}で考察しよう.}.
  
  全称記号$\forall$と存在記号$\exists$とを総称して
  \index[widx]{げんていきごう@限定記号 \, quantifier}
  \emph{限定記号}(quantifier)と呼ぶ.
  また,限定記号$\forall,  \exists$を用いる体系を
  \index[widx]{じゅつご@述語 \, prodicate!ろんり@---論理 \, --- logic}
  \emph{述語論理}(predicate logic)と呼ぶ
  \footnote{より正確には,この資料で展開する述語論理は一階述語論理と呼ばれている.}
  .
  これに対して命題結合記号$\lnot ,  \land ,  \lor ,  \to$
  を用いる体系を\index[widx]{めいだい@命題 \, proposition!ろんり@---論理 \, propositional logic}
  \emph{命題論理}(propositional logic)という.

  もっとも,$\forall ,  \exists$だけを切り離し,
  他の論理記号と区別して研究してもあまり意味はないので,
  述語論理においては結局のところ他の論理記号をすべて取り扱うことになる.
  従って,述語論理は命題論理を含んだ体系であると考えるのが普通である.

  命題結合記号同士には結合の優先順位を定めたが,
  限定記号に関しては,限定記号はどの命題結合記号よりも結合が強い,
  あるいは$\lnot$と結合の強さが同じであると約束しておく.
  直感的にも納得がいくであろう.

  
 \paragraph{矛盾記号}
  命題$A$の否定$\lnot A$とは,「$A$が間違いである」という意味である.
  では,$A$が間違いであるということをどうやって示せばよいかといえば,
  $A$を仮定すると「矛盾する」ということを導くのが一般的であろう.
  この「矛盾」というのを
  \begin{align}
    \curlywedge
    \label{eq:mujun}
  \end{align}
  と表す.これは1つの「間違った命題」もしくは「偽な命題」
  とでも思っておけばよい.

  以下,この資料で展開する命題論理と述語論理においては,
  矛盾記号$\curlywedge$も取り扱うことにする.

  さて,変数$x$の条件$F(x)$について,命題$\forall x F(x)$が
  偽である,すなわち$\forall x F(x)$を仮定すると矛盾する
  ことを示すには,$\lnot F(a)$となる$a$が存在する,
  すなわち命題$\exists x \lnot F(x)$が真であることを示せばよい.
  これを示すには,$\lnot F (a)$となるような$a$を具体的に用意してやるのが
  手っ取り早い.
  このような$a$のことを
  全称命題$\forall x F(x)$の\index[widx]{はんれい@反例 \, counter-example}
  \emph{反例}(counter-example)
  という.

  \begin{ex}
    $x$を実数を表す変数であるとして,
    命題「$\forall x ( x \leq 0)$」は偽である.
    反例として,たとえば$x=1$がとれる(他にも無数にとれる).
  \end{ex}

  %%%%%%%%%%%%%%%==================演習問題=====================%%%%%%%%%%%%%%%%
  \begin{que} \label{chp:sequent.sec:ronri.que:singihantei}
      次の命題の真偽を判定せよ.
      ただし,全称命題が偽である場合には反例を挙げよ.
        \begin{enumerate}
          \item $\lnot ( 1=2 \to 2=2)$ % 偽 
          \item $(1=2 \to 3=2) \lor ( 1=1 \land 1=3)$ % 真 
          \item $x$は実数を表す変数であるとして$\forall x  \lnot ( x^2+x+1 \leq 0 )$ % 真
          \item $x,  y,  z$は自然数を表す変数だとして
                $\forall x, y \exists z ( x + z =y)$ % 偽 反例 x=3 , y =2
          \item $(2=0 \to 1=1) \to 2=3$ % 偽
          \item $2=0 \to ( 1=1 \to 2=3)$ % 真
        \end{enumerate}
  \end{que}
  %%%%%%%%%%%=================演習問題終わり====================%%%%%%%%%%%%%
  %
  %
 \section{変数の束縛}
 \label{sec:hensuu}

\paragraph{自由変数と束縛変数}
 $x,  y,  z$を自然数を表す変数として,
 $x,  y,  z$の条件$F(x,y,z)$を「$x+y=z$」としよう.
 このとき,存在命題$\exists x F(x,y,z)$,すなわち$\exists x ( x + y =z)$
 は,内容としては「$y < z$」を意味していることになる
 \footnote{この資料では自然数は0を含まないとしているので$<$であるが,
 自然数は0を含めるという立場を取るならば$\leq$とするのが正しい.}.
 ここで重要なのは,文中の$x$はその真偽に影響せず,
 $\exists x F(x,y,z)$の真偽は$y,  z$によってのみ決定されることである.
 存在命題$\exists x F(x,y,z)$において,$y,  z$は値を代入することのできる
 いわば「本当の変数」であるが,$x$は内容に関与しない「見かけ上の変数」である.

 命題(条件)において,限定記号$\forall ,  \exists$とともに用いられている
 「見かけ上の変数」を\index[widx]{そくばくへんすう@束縛変数 \, bound variable}
 \emph{束縛変数}(bound variable)といい,
 値を代入することができる「本当の変数」を
 \index[widx]{じゆうへんすう@自由変数 \, free variable}
 \emph{自由変数}(free variable)という.
 プログラミングにおけるグローバル変数とローカル変数の関係と似ている.

 束縛変数は記号を別のものに変えても「内容」は変わらない.
 $\exists x F(x,y,z)$と書こうが$\exists t F(t,y,z)$と書こうが同じことである.
 ただし,$\exists t F(x, y,z)$と書いたら意味は変わってしまう.
 重要なのは記号間の対応であり,
 その記号が何と書かれているかは「内容」には関わらないのである.

 \begin{ex}
   2変数関数$f(x,y)=xy^2$について,$f(x,y)$を$1 \leq x \leq 3$まで積分すると,
   $y$の1変数関数が得られる.これを$g(y)$とおくと,
   \begin{align*}
     g(y) = \int_1^3 xy^2 \, dx = \left[ \frac{1}{2} x^2y^2 
     \right]_{x=1}^{x=3} = 4y^2
   \end{align*}
   となる.積分変数を$t$に変えても
   \begin{align*}
     \int_{1}^3 t y^2 \, dt = \left[ \frac{1}{2} t^2 y^2
     \right]_{t=1}^{t=3} = 4y^2 =g(y)
   \end{align*}
   と,得られる関数は変わらない.しかし,束縛変数と同じ記号を代入してしまうと
   \begin{align*}
     g(x) = \int_1^3 x x^2 dx = \left[ \frac{1}{4} x^4 
     \right]_{x=1}^{x=3} = 20 \neq \int_1^3 t x^2 dt = 6x^2
   \end{align*}
   と,おかしな結果になる.
 \end{ex}

 どうしてこのようなおかしな結果が導かれたかといえば,
 積分変数として仮においていただけの$x$と,関数の独立変数としての$y$とを
 ごっちゃにしてしまったからだ.

 こうした不都合を回避するためには,「仮におく変数」と「真の変数」,
 すなわち束縛変数と自由変数をきちんと区別し,記号を使い分ければよい.
 もちろんどの記号が自由変数を表していて,どの記号が束縛変数を表しているかは
 文脈によってまちまちである.しかし,束縛変数の方は限定記号や積分記号のように
 わかりやすい目印があるだろうから参考にしてほしい.

 さて,束縛変数は「仮の変数」であり,「内容」には関わらないのであった.
 束縛変数はある特定の範囲でのみ「意味」を持ち,
 その範囲から外では存在しないものとみなされる.
 従って,束縛変数がどこまで「意味」を持つのか,
 すなわち束縛変数の作用範囲を明示しなければ誤解が生じてしまう.

 \begin{ex}
   \begin{align*}
     \sum_{k=1}^{n} ( k^2 - k) & = \sum_{k=1}^{n} k^2 - \sum_{k=1}^{n} k \\
     & = \frac{1}{6} n(n+1)(2n+1) - \frac{1}{2} n(n+1)
   \end{align*}
   であるが,
   \begin{align*}
     \sum_{k=1}^{n} k^2 - k = \frac{1}{6} n (n+1)(2n+1) - k 
     \neq \sum_{k=1}^{n} ( k^2 -k)
   \end{align*}
   である.
 \end{ex}
 
 察しろと言いたくなるような例ではあるものの,
 何らかの手段(普通はカッコ)を用いて束縛変数の作用範囲を
 明示してやらないと意図しない解釈がなされてしまうのである.
 とはいえ,これは悪いことばかりではない.
 
 \begin{ex}
   $x$は実数を表す変数として,
   命題$\forall x ( {x \leq 0} \lor {x > 0})$は真である.
   しかし,命題$\forall x (x \leq 0 ) \lor \forall x (x >0) $
   は偽である.
 \end{ex}
 
 束縛変数が作用範囲の外では何の意味も持たないことを利用すれば,
 このように表記を美しくしたり,使う記号の数を節約したりできるのである.

 束縛変数の作用範囲について考えてみれば,ちまたでよく見る
 \[
   \forall x : F(x) \to A 
 \]
 やら
 \[
   \forall \varepsilon > 0 , \;  \exists N \;\; s.t. \;\; \forall n , \; n \geq N \Longrightarrow 
   \lvert a_n - \alpha \rvert < \varepsilon
 \]
 なんていう記法がよろしくないことはすぐにわかる.
 もちろん,文章中で「$\forall x$に対して」だったり「$\exists x$なので」
 などと書くのもご法度である.
 論理記号は単に文章を記号化したものではない.
 もちろん,論理記号で書かれた記号列を文章に翻訳する,
 もしくはその逆を実行することはできるが,
 それは「論理記号で書かれた記号列」と
 「論理記号で書かれた記号列を文章に翻訳したもの」は等価である
 ことを意味しているわけではない.
 あくまで,「論理記号で書かれた文字列」に人間が「勝手に意味を付け加えている」
 のである
 \footnote{とはいえ,論理記号の使い方に関して重箱の隅をつつくように騒ぎ立てるのも考えものである.
   しかし,記号論理を学んでいるときには,
 重箱の隅をつつくように気をつけるくらいでちょうどいいと思う.}
 .

 \paragraph{対象領域}
 変数を含んだ命題や条件を考えるとき,
 「$x$は自然数を表す変数だとして」のように,
 変数が動きうる範囲をあらかじめ指定しておくことがあった.

 この「変数が動きうる領域」が
 集合である場合,その集合のことをその変数の
 \index[widx]{たいしょうりょういき@対象領域 \, domain}
 \emph{対象領域}(domain)と呼ぶ
 \footnote{集合とはいっても,対象領域が空である場合は考えない.}.
 変数$x$の対象領域が集合$D$である場合,
 「$\forall x ( x \in D \to (\cdots x \cdots) )$」という形の命題(条件)は
 「$D$に属するすべての$x$に対して$\cdots x \cdots$」
 という意味であるから,これを「$\forall x \in D ( \cdots x \cdots)$」
 と略記することにしよう.同様に,
 「$\exists x \in D ( \cdots x \cdots )$」という形の命題(条件)は
 「$\exists x ( x \in D \land ( \cdots x \cdots ))$」
 の略記であるということにしておく.
 %
 %
 %%%%%%%%%%%%%%%===============演習問題===============%%%%%%%%%%%%
 \begin{que} \label{chp:sequent.sec:hensuu.que:sokubakusingi}
   次の命題の真偽を判定せよ.ただし,全称命題が偽である場合には反例を挙げ,
   存在命題が真である場合には例を挙げよ.
  \begin{enumerate}
     \item $\forall x \in \mathbb{R} ( x \leq 2 \to x \leq 0)$ % 偽 反例x=1
     \item $\forall x \in \mathbb{R} ( x \leq 2 ) % 真 全称命題ではない
       \to \forall x \in \mathbb{R} ( x \leq 0)$ % 真
     \item $\exists x \in \mathbb{R} ( x \leq 2 \land 2 < x)$ % 偽
     \item $\exists x \in \mathbb{R} ( x \leq 2) 
            \land \exists x \in \mathbb{R} ( 2 < x)$ % 真
     \item $\exists x \in \mathbb{R} ( x \leq 2 \to 1=2)$ % 真 x = 3
     \item $\exists x \in \mathbb{R} ( x \leq 2 ) \to 1=2 $ % 偽
   \end{enumerate}
 \end{que}
 %
 %
 %
 \section{必要性と十分性}
 \label{sec:hituyoujubun}
 %
 %

 \paragraph{必要条件と十分条件}
 
 $x$の条件$F(x),  G(x)$に対し,
 命題
 \begin{align}
   \forall x ( F(x) \to G(x))
   \label{eq:FnarabaG}
 \end{align}  
 が真であるとき,
 $F(x)$は$G(x)$であるための
 \index[widx]{じょうけん@条件 \, condition!じゅうぶん@十分--- \, sufficient ---}
 \emph{十分条件}(sufficient condition)であるといい,
 $G(x)$は$F(x)$であるための
 \index[widx]{じょうけん@条件 \, condition!ひつよう@必要--- \, necessary ---}
 \emph{必要条件}(necessary condition)であるという.

 \begin{ex} \label{chp:sequent.sec:hituyoujubun.ex:xleq}
   命題$\forall x \in \mathbb{R} ( x \leq 1 \to x\leq 2)$
   は真である.従って,実数$x$について,
   $x \leq 1$であることは$x \leq 2$であるための
   十分条件で,$x \leq 2$であることは$x \leq 1$であるための
   必要条件である.
 \end{ex}
 なお,高校の教科書では,$x$の条件$F(x),  G(x)$に対し,
 命題$\forall x (F(x) \to G(x) )$を
 \begin{align}
   F(x) \Longrightarrow G(x)
   \label{eq:narabakoukou}
 \end{align}
 と書き表すことがある.
 この資料で用いる「$\Longrightarrow$」とは意味がまったく異なるので注意されたい.

 例\ref{chp:sequent.sec:hituyoujubun.ex:xleq}を眺めてみれば
 必要・十分のネーミングの意味がわかるはずだ.
 $x \leq 1$が成り立つためには$x \leq 2$が成り立っていなければならない.
 すなわち,$x \leq 1$となるためには$x \leq 2$であることが「必要」なのであり,
 $x \leq 2$が成り立つためには$x \leq 1$が成り立っていればよい.
 すなわち,$x \leq 2$となるためには$x \leq 1$であれば「十分」なのである.

 2つの$x$に関する条件$F(x),  G(x)$に対し,
 命題
 \begin{align}
   \forall x ( F(x) \rightleftarrows G(x))
   \label{eq:FGdouti}
 \end{align}
 が真である,
 すなわち命題
 \begin{align}
   \forall x( (F(x) \to G(x) ) \land ( F(x) \to G(x) ))
   \label{eq:FGdouti2}
 \end{align}
 が真であるとき,$F(x)$は$G(x)$であるための
 \index[widx]{じょうけん@条件 \, condition!ひつようじゅうぶん@必要十分--- 
 \quad necessary and sufficient ---}
 \emph{必要十分条件}(necessary and sufficient condition)であるという.

 \begin{ex}
   命題$\forall x \in \mathbb{R} ( x= 1 \lor x = -1 \rightleftarrows x^2 =1)$
   は真である.従って,実数$x$について,
   $x =1 \lor x=-1$であることは$x^2=1$であるための必要十分条件である.
 \end{ex}

 高校の教科書では,2つの$x$に関する条件$F(x) , G(x)$について,
 命題$\forall x (F(x) \rightleftarrows G(x))$のことを
 \begin{align}
   F(x) \Longleftrightarrow G(x)
   \label{eq:FGkoukoudouti}
 \end{align}
 と表すことがある.
 
 必要条件,十分条件は変数の数が増えても同様に定義される.
 $x,y$に関する2つの条件$F(x,y),G(x,y)$に対し,
 命題
 \begin{align}
   \forall x,y ( F(x,y) \to G(x,y) )
   \label{eq:FnarabaGxy}
 \end{align}
 が真であるとき,$F(x,y)$は$G(x,y)$であるための十分条件であるといい,
 $G(x,y)$は$F(x,y)$であるための必要条件であるという.さらに,
 $x,y$に関する2つの条件$F(x,y) , G(x,y)$に対し,
 命題
 \begin{align}
   \forall x,y ( F(x,y) \rightleftarrows G(x,y))
   \label{eq:xyFGdouti}
 \end{align}
 が真であるとき,$F(x,y)$は$G(x,y)$であるための必要十分条件であるという.

 一般の場合も同様である.
 $n$個の変数$x_1, x_2, \ldots , x_n$に関する2つの条件
 $F(x_1, \ldots , x_n), G(x_1, \ldots , x_n)$について,
 命題
 \begin{align}
   \forall x_1, \ldots , x_n ( F(x_1, \ldots , x_n ) \to G(x_1, \ldots , x_n))
   \label{eq:nFGnaraba}
 \end{align}
 が真であるとき,$F(x_1, \ldots x_n)$は$G(x_1, \ldots , x_n)$
 であるための十分条件であるといい,
 $G(x_1, \ldots , x_n)$は$F(x_1, \ldots , x_n)$であるための必要条件であるという.
 さらに,$n$個の変数$x_1, x_2, \ldots , x_n$に関する2つの条件
 $F(x_1, \ldots , x_n), G(x_1, \ldots , x_n)$について,
 命題
 \begin{align}
   \forall x_1, \ldots , x_n ( F(x_1, \ldots , x_n ) \rightleftarrows G(x_1, \ldots , x_n))
   \label{eq:nFGnarabadouti}
 \end{align}
 が真であるとき,$F(x_1, \ldots x_n)$は$G(x_1, \ldots , x_n)$
 であるための必要十分条件であるという.


 \begin{que} \label{chp:sequent.sec:hituyoujubun.que:xhituyoujubun}
   次の各主張が正しいかどうか判定せよ.
   \begin{enumerate}
     \item $x$に関する2つの条件$F(x), G(x)$について,
       「$F(x)$は$G(x)$であるための必要十分条件である」とき,
       $G(x)$は$F(x)$であるための必要条件ではない. % 正しくない
     \item 2つの$x$に関する条件$F(x),G(x)$について,「$F(x)$であるための必要条件は$G(x)$である」
       ことを示すには,「$F(x)$を満たすすべての$x$が$G(x)$を満たす」
       ことを示せばよい. % 正しい
     \item 2つの$x$に関する条件$F(x),G(x)$に対し,
       「$G(x)$を満たす$x$で$F(x)$を満たすものはない」ことが示されたとき,
       「$\lnot F(x)$は$G(x)$であるための必要条件である」ことがいえる.% 正しい
   \end{enumerate}
 \end{que}

 \begin{que} \label{que:kigoukaranihongo}
   次の記号列を言葉に翻訳せよ.ただし,「$\forall x >0(\cdots x \cdots )$」は
   「$\forall x \in \mathbb{R}( x>0 \to (\cdots x \cdots ))$」を略記したものであり,
   「$\exists x>0 (\cdots x \cdots )$」は「$\exists x \in \mathbb{R}
   ( x>0 \land ( \cdots x \cdots ))$」
   の略記である.
   \begin{enumerate}
     \item $\exists x F(x) \land \forall x,y (F(x) \land F(y) \to x=y)$
     \item $\forall \varepsilon >0 \exists \delta >0 \forall x \in I
       ( 0< \lvert x- a \rvert < \delta \to \lvert f(x) - A \rvert < \varepsilon)$
     \item $\forall y \in Y \exists x \in X (y=f(x))$
     \item $\forall x_1, x_2 \in X (f(x_1) = f(x_2) \to x_1 = x_2)$
     \item $\forall a,b >0 \exists N \in \mathbb{N} ( Na > b)$
   \end{enumerate}
 \end{que}

 \begin{que} \label{que:nihongokarakigou}
   次の文章を論理記号を用いて表現せよ.
   \begin{enumerate}
     \item $x \in A$であれば必ず$x \in B$となる.
     \item 任意の正の数$\varepsilon$に対して$N \in \mathbb{N}$が存在して,
       $n, m \geq N$を満たす任意の$n,m \in \mathbb{N}$について
       $\lvert a_n - a_m \rvert < \varepsilon$となる.
     \item $x^2+y^2=1$を満たす任意の実数$x,y$に対し,
       $x= \cos \theta $と$ y= \sin \theta$をともに満たす実数$\theta$で
       $0 \leq \theta < 2 \pi$となるものが存在する.
     \item 任意の$x \in S$に対して$x \leq M$を成り立たせるような
       実数$M$のなかで最小のものが存在する. 
   \end{enumerate}
 \end{que}

 %
 %
 %
 %
 \section{シークエントを利用した自然演繹}
 \label{sec:sequent}
 %
 数学では,基礎となる公理を出発点とし,そこから演繹的に証明を重ねることによって
 数々の定理を導き出している.
 この節では,数学において通常行われている演繹的推論について,
 そこからいくつかの規則を抽出し,
 証明の形式化を試みる.

 \paragraph{シークエント}
 命題(条件)の有限列$A_1, A_2, \ldots , A_n$
 および命題(条件)$B$について,
 「$A_1, A_2, \ldots , A_n$を仮定すると$B$を導くことができる」ことを
 \begin{align}
   A_1 , A_2, \ldots , A_n \Longrightarrow B
   \label{eq:sequent}
 \end{align}
 と書き表し,$A_1, A_2 , \ldots ,A_n $を左辺(前提),
 $B$を右辺(結論)とする
 \index[widx]{しーくえんと@シークエント \, sequent}
 \emph{シークエント}(sequent)という.
 ただし,$n=0$の場合として,
 \begin{align}
   \qquad \Longrightarrow B
   \label{eq:sequentempty}
 \end{align}
 という形式のシークエントも許し,「前提なしで$B$が成り立つ」
 ということを表すものと約束する.

 \begin{ex}
   $a,  b,  c$を実数として,
   \begin{align*}
    a>0 , b^2-4ac \geq 0 \Longrightarrow 
    \exists x \in \mathbb{R} ( ax^2 +bx+c=0)
   \end{align*}  
   は$a,  b,  c$に関する条件$a>0$と$b^2 - 4ac \geq 0$を前提とし,
   条件$\exists x \in \mathbb{R} ( ax^2+bx+c=0)
   $を結論とするシークエントである.
   一方,
   \begin{align*}
     a>0 \land b^2 -4ac \geq 0 \to
     \exists x \in \mathbb{R} ( ax^2 +bx +c =0)
   \end{align*}
   は$a,  b,  c$についての1つの条件である.
 \end{ex}

 シークエントの前提に現れる命題(条件)の有限列を
 $\varGamma$や$\varDelta$といったギリシャ文字の大文字で表すことが多い.
 ただし,列とはいってもその列に現れる命題(条件)の
 種類のみを考え,重複や順序は気にしないことにする.
 また,命題(条件)の有限列$\varGamma,  \varDelta$について,
 $\varGamma$に現れる命題(条件)は必ず$\varDelta$にも現れることを
 $\varGamma \subset \varDelta$と表すことにする.

 \begin{ex}
   $A,  B,  C,  D$を命題(条件)として,$\varGamma$が$A,  C,  B$
   という列を表し,$\varDelta$が$D,  B$という列を表すとする.
   このとき,$\varGamma,  \varDelta$という前提を考えるとき,
   $A,  C,  B,  D,  B$ではなく$A,  B,  C,   D$
   と扱ってよい.
 \end{ex}

 命題(条件)$A$に対し,
 \begin{align}
   A \Longrightarrow A 
   \label{eq:sisiki}
 \end{align}
 の形のシークエントを$A$についての(論理的)
 \index[widx]{ししき@始式 \, initial sequent}
 \emph{始式}(initial sequent)という.
 「$A$を仮定すれば$A$が導ける」という意味のシークエントである.

 \paragraph{推論規則}
 あるシークエントが与えられたとき,
 そのシークエントから
 別のシークエントを導くためのルールのことを
 \index[widx]{すいろんいきそく@推論規則 \, inference rule}
 \emph{推論規則}(inference rule)という.
 どのような推論規則を採用するかについてはたくさんの流儀がある.
 ここでは,シークエント計算を用いた古典論理の自然演繹と呼ばれる体系の
 推論規則を見ていく
 \footnote{この資料で挙げる体系は,一般に「LK」として知られているものとは
 形が相当異なるものである.この体系は参考文献の\cite{yosida}がもとになっている.}
 .
 
 推論規則は,$A,  B,  C,  D,  E,  F$を命題(条件)として,
  \begin{prooftree}
        \AxiomC{$A \Longrightarrow B$}
        \AxiomC{$C \Longrightarrow D$}
     \LeftLabel{(規則名)}
        \BinaryInfC{$E \Longrightarrow F$}
  \end{prooftree}
 という形式で記述される.
 これは「もしシークエント$A \Longrightarrow B$と
 $C \Longrightarrow D$がともに成り立つであれば,
 シークエント$E \Longrightarrow F$が成り立つ」という意味である.
 上側にあるシークエントの数が変わっても同様に解釈してほしい. 

 この資料では,以下に示すような推論規則を採用する.
 \begin{oframed}
  $A,  B,  C$は任意の命題(条件)
  $\varGamma ,  \varDelta$は命題(条件)の任意の有限列(空列でもよい),
  $F(a)$は自由変数$a$に関する任意の条件,$t$は任意の項とし,
  $\curlywedge$は「矛盾」を表す命題とする.
  変数条件として,
  $\forall$導入においては自由変数$a$は$\varGamma,  \forall xF(x)$には現れず,
  $\exists$除去においては自由変数$a$は$\varGamma,  \exists x F(x),  C$
  には現れないものとする:

  \begin{spacing}{2}
        \AxiomC{$\varGamma \Longrightarrow B$}
     \LeftLabel{前提の増加}
     \RightLabel{(ただし$\varGamma \subset \varDelta$とする)}
        \UnaryInfC{$\varDelta \Longrightarrow B$}
   \DisplayProof
    %
        
    %
    \quad
        \AxiomC{$\varGamma \Longrightarrow A $}
        \AxiomC{$\varGamma \Longrightarrow B $}
     \LeftLabel{$\land$導入}
        \BinaryInfC{$\varGamma \Longrightarrow A \land B$}
   \DisplayProof \\
    %
        \AxiomC{$\varGamma \Longrightarrow A \land B$}
     \LeftLabel{$\land$除去(左)}
        \UnaryInfC{$\varGamma \Longrightarrow A$}
   \DisplayProof 
    %
    \quad
        \AxiomC{$\varGamma \Longrightarrow A \land B $}
     \LeftLabel{$\land$除去(右)}
        \UnaryInfC{$\varGamma \Longrightarrow B$}
   \DisplayProof
    %
        \AxiomC{$\varGamma \Longrightarrow A$}
     \LeftLabel{$\lor$導入(左)}
        \UnaryInfC{$\varGamma \Longrightarrow A \lor B$}
   \DisplayProof
    %
    \quad
        \AxiomC{$\varGamma \Longrightarrow B$}
     \LeftLabel{$\lor$導入(右)}
        \UnaryInfC{$\varGamma \Longrightarrow A \lor B$}
   \DisplayProof
    %
        \AxiomC{$\varGamma \Longrightarrow A \lor B$}
        \AxiomC{$A,  \varGamma \Longrightarrow C$}
        \AxiomC{$B,  \varGamma \Longrightarrow C$}
     \LeftLabel{$\lor$除去}
        \TrinaryInfC{$\varGamma \Longrightarrow C$}
   \DisplayProof
    %
        \AxiomC{$A,  \varGamma \Longrightarrow \curlywedge$}
     \LeftLabel{$\lnot$導入}
        \UnaryInfC{$\varGamma \Longrightarrow \lnot A$}
   \DisplayProof
    %
    \quad
        \AxiomC{$\varGamma \Longrightarrow A$}
        \AxiomC{$\varGamma \Longrightarrow \lnot A$}
     \LeftLabel{$\lnot$除去}
        \BinaryInfC{$\varGamma \Longrightarrow \curlywedge$}
   \DisplayProof
    %
        \AxiomC{$\varGamma \Longrightarrow \lnot \lnot A$}
     \LeftLabel{2重否定の除去}
        \UnaryInfC{$\varGamma \Longrightarrow A$}
   \DisplayProof 
    %
        \AxiomC{$A ,  \varGamma \Longrightarrow B$}
     \LeftLabel{$\to$導入}
        \UnaryInfC{$\varGamma \Longrightarrow A \to B$}
   \DisplayProof
    %
        \AxiomC{$\varGamma \Longrightarrow A$}
        \AxiomC{$\varGamma \Longrightarrow A \to B$}
     \LeftLabel{$\to$除去(modus ponens)}
        \BinaryInfC{$\varGamma \Longrightarrow B$}
   \DisplayProof
    \\ 
   %
        \AxiomC{$\varGamma \Longrightarrow F(a)$}
     \LeftLabel{$\forall$導入}
        \UnaryInfC{$\varGamma \Longrightarrow \forall x F(x)$}
   \DisplayProof
    \quad 
   %
        \AxiomC{$\varGamma \Longrightarrow \forall x F(x)$}
     \LeftLabel{$\forall$除去}
        \UnaryInfC{$\varGamma \Longrightarrow F(t)$}
   \DisplayProof
    %
        \AxiomC{$\varGamma \Longrightarrow F(t)$}
     \LeftLabel{$\exists$導入}
        \UnaryInfC{$\varGamma \Longrightarrow \exists x F(x)$}
   \DisplayProof \\
    %
        \AxiomC{$\varGamma \Longrightarrow \exists x F(x) $}
        \AxiomC{$F(a),  \varGamma \Longrightarrow C$}
     \LeftLabel{$\exists$除去}
        \BinaryInfC{$\varGamma \Longrightarrow C$}
   \DisplayProof
  \end{spacing}  
 \end{oframed}
 $\forall$導入と$\exists$除去には変数条件が課されている.
 $\forall$導入の推論規則
 \begin{prooftree}
        \AxiomC{$\varGamma \Longrightarrow F(a)$}
   \LeftLabel{$\forall$導入}
        \UnaryInfC{$\varGamma \Longrightarrow \forall x F(x)$}
 \end{prooftree}
 に課された変数条件は「自由変数$a$は$\varGamma ,  \forall x F(x)$に現れないこと」
 である.推論規則としての意味は「任意の自由変数$a$に対し,
 $\varGamma$から$F(a)$が導けたのであれば,$\varGamma$から$\forall x F(x)$が導ける」
 という意味であるから,$\varGamma$と$\forall x F(x)$に$a$が
 現れてはいけないというのはある意味で当然ともいえる.

 $\forall$導入の推論規則は「$\varGamma$から$\forall x F(x)$を導きたいとき,
 代わりに自由変数$a$を任意にとり,その$a$に対して$\varGamma$から$F(a)$
 を導いてよい」というように解釈することができる.
 $\forall x F(x)$は「すべて(all)の$x$に対して$F(x)$」という意味であり,
 自由変数$a$を任意にとった場合の$F(a)$というのは
 「任意の(arbitrary)$a$に対して$F(a)$」という意味である.
 両者のニュアンスは少し異なる.
 前者は「無限に多くある変数$x$を一挙に見てそのすべてが$F(x)$を満たす」
 というニュアンスであるが,後者は
 「無限に多くある変数$x$の中から$a$を1つ任意に選び出し,
 その1つの$a$が$F(a)$を満たす」
 というニュアンスである.
 前者から後者を結論してよいという推論規則が$\forall$除去であり,
 後者から前者を結論してよいという推論規則が$\forall$導入である.
 無限に多くある$x$のすべてが$F(x)$を満たすことを示すのは
 有限の時間しか生きられない我々には不可能である.
 しかし,任意に1つとった自由変数$a$が$F(a)$を満たすことを示すのは
 有限の時間でも可能である.
 両者を別物として考えてしまうと数学理論を構築することがとても困難になってしまう.
 よく``数学においては「すべて」と「任意」は区別しない''
 と言われるのにはこういう事情があるのである.

 $\exists$除去の推論規則
 \begin{prooftree}
        \AxiomC{$\varGamma \Longrightarrow \exists x F(x)$}
   \LeftLabel{$\exists$除去}
        \AxiomC{$F(a) ,  \varGamma \Longrightarrow C$}
        \BinaryInfC{$\varGamma \Longrightarrow C$}
 \end{prooftree}
 にも「自由変数$a$は$\varGamma ,  \exists x F(x),  C$には現れない」
 という変数条件が課されている.
 推論規則の意味としては「$\varGamma$から$F(x)$を満たす$x$が存在することがわかったとする.
 $F(x)$を満たす$x$を1つとり,仮に$a$という名前をつけたとして,
 $a$がたとえどんなものであったとしても$F(a)$と$\varGamma$から
 $C$が導けたとするならば,$\varGamma$だけから$C$を導ける」
 という意味であるからこの変数条件にも納得がいくであろう.


 \paragraph{シークエント計算}
 我々がいま導入した推論規則は,
 命題結合記号や限定記号の「意味」を考えればごく「自然」なものである.

 ここからは,命題結合記号や限定記号の「意味」は忘れ,
 これらの推論規則のみをもとにしてシークエントを導出し,
 演繹的推論を形式的な記号操作として実行することを考える.
 
 命題(条件)の有限列$\varGamma$と命題(条件)$A$について,
 シークエント$\varGamma \Longrightarrow A$が
 \index[widx]{どうしゅつかのう@(シークエントが)導出可能 \, derivable}
 \emph{導出可能}(derivable)であるとは
 \begin{enumerate}
   \item $\varGamma \Longrightarrow A$が始式の形をしている.
   \item すでに導出可能とわかっているいくつかのシークエントに対して
         推論規則を適用することで$\varGamma \Longrightarrow A$が得られる.
 \end{enumerate}
 のいずれかが成り立つことをいう.
 推論規則においては始式については言及されなかったが,
 始式は必ず導出可能であることを認めるのである.
 これは,始式を公理として採用したことに相当する.

 命題(条件)$A$について,シークエント
 \begin{align*}
   \Longrightarrow A
 \end{align*}
 が導出可能であることを,$A$は
 \index[widx]{しょうめいかのう@(命題が)証明可能 \, provable}
 \emph{証明可能}(provable)であるという.

 また,2つの命題(条件)$A,  B$について,
 2つのシークエント$A \Longrightarrow B$と
 $B \Longrightarrow A$がともに導出可能であるとき,
 $A$と$B$は\index[widx]{どうちである@同値である \, equivalent}
   \emph{同値}(equivalent)であるといい,
 \begin{align}
   A \equiv  B
 \end{align}
 と書き表す
 \footnote{「$\Longleftrightarrow$」という記号が使いたくなるが,
 この資料で言及しないもろもろの事情により,
 「$\Longleftrightarrow$」という記号はここでは用いない.
 「$\Longleftrightarrow$」という記号は日本語の「同値である」
 の略記であると思っておけばよい.
 }
 .

 この資料では,命題結合記号「$\to$」とシークエントを構成する記号「$\Longrightarrow$」を
 異なる記号として導入した.
 しかし,「意味」としてはそれほど変わるわけではない.
 それを象徴するのが次の定理である.
 
 \begin{thm}[演繹定理] \label{thm:toarrowequiv}
   命題(条件)$A,  B$について,
   シークエント$A \Longrightarrow B$
   が導出可能であるための必要十分条件は,
   命題(条件)$A \to B$が証明可能であることである.
 \end{thm}
 \begin{proof}(必要性)シークエント$A \Longrightarrow B$が導出可能であるとする.
   このとき,
   \begin{prooftree}
                              \AxiomC{仮定}
                              \noLine
                              \UnaryInfC{$A \Longrightarrow B $}
                         \LeftLabel{$\to$導入}
                              \UnaryInfC{$\Longrightarrow A \to B$}
   \end{prooftree}
   よって,$A \to B$は証明可能である. \\
   (十分性)$A \to B$が証明可能であると仮定する.
   このとき,
   \begin{prooftree}
                              \AxiomC{仮定}
                              \noLine
                              \UnaryInfC{ $\Longrightarrow A \to B$}
                              \UnaryInfC{$A \Longrightarrow A \to B$}
           \AxiomC{始式}
           \noLine
           \UnaryInfC{$A \Longrightarrow A$}
                          \LeftLabel{$\to$除去}
                          \BinaryInfC{$A \Longrightarrow B$}
   \end{prooftree}
   従って,シークエント$A \Longrightarrow B$は導出可能である.
 \end{proof}
 定理\ref{thm:toarrowequiv}の証明のように,
 「前提の増加」以外の推論規則については
 自分が何の推論規則を用いたかを書き記しておくのがお約束である.

 また,定理\ref{thm:toarrowequiv}により,以下のことはすぐにわかる:

 \begin{coro}
   命題(条件)$A,  B$について,$A$と$B$が同値であるための必要十分条件は,
   命題(条件)$A \rightleftarrows B$が証明可能であることである.
 \end{coro}
 
 命題(条件)の同値を表す記号$\equiv$は次の関係を満たす.
 証明は容易であろう.
 \begin{lemma}
   命題(条件)$A,  B,  C$について,次の関係が成り立つ.
   \begin{enumerate}[1. ]
     \item $A \equiv A$となる.
     \item $A \equiv B$ならば$B \equiv A$となる.
     \item $A \equiv B$かつ$B \equiv C$ならば,$A \equiv C$となる.
   \end{enumerate}
 \end{lemma}
 「矛盾からは何でも導ける」というのはよく一般に言われることであるが,
 それを象徴するのが次の補題\ref{lemma:mujun}である.
 \begin{lemma}[矛盾に関する推論法則] \label{lemma:mujun}
    $\varGamma$を命題(条件)の有限列,$C$を任意の命題(条件)とする.
    このとき,矛盾に関する推論法則
    \begin{prooftree}
      \AxiomC{$\varGamma \Longrightarrow \curlywedge$}
      \UnaryInfC{$\varGamma \Longrightarrow C$}
    \end{prooftree}
    が成り立つ.
 \end{lemma}
 \begin{proof}
   シークエント$\varGamma \Longrightarrow \curlywedge$が導出可能であると仮定して,
   シークエント$\varGamma \Longrightarrow C$が導出可能であることを示せばよい.
  \begin{prooftree}
     \AxiomC{仮定}
     \noLine
     \UnaryInfC{$\varGamma \Longrightarrow \curlywedge$}
     \UnaryInfC{$\lnot C,  \varGamma \Longrightarrow \curlywedge$}
     \LeftLabel{$\lnot$導入}
     \UnaryInfC{$\varGamma \Longrightarrow \lnot \lnot C$}
     \RightLabel{2重否定の除去}
     \UnaryInfC{$\varGamma \Longrightarrow C$}
  \end{prooftree}
   % \DisplayProof
   従って,シークエント$\varGamma \Longrightarrow \curlywedge$が導出可能であれば,
   シークエント$\varGamma \Longrightarrow C$も導出可能である.
 \end{proof}
 今後,矛盾に関する推論法則も用いることにする.

 推論法則として有名なものをもうひとつ導いておこう.
 \begin{lemma}[cut規則]
   $\varGamma, \varDelta$を任意の命題(条件)の有限列とし,
   $A,B$を任意の命題(条件)とする.このとき,cut規則
   \begin{prooftree}
     \AxiomC{$\varGamma \Longrightarrow A$}
     \AxiomC{$A, \varDelta \Longrightarrow B$}
     \BinaryInfC{$\varGamma , \varDelta \Longrightarrow B$}
   \end{prooftree}
   が成り立つ.
 \end{lemma}

 \begin{proof}
   2つのシークエント$\varGamma \Longrightarrow A$と
   $A, \varDelta \Longrightarrow B$がともに導出可能であるとすると
   \begin{prooftree}
     \AxiomC{仮定}
     \noLine
     \UnaryInfC{$\varGamma \Longrightarrow A$}
     \UnaryInfC{$\varGamma , \varDelta \Longrightarrow A$}
     \AxiomC{仮定}
     \noLine
     \UnaryInfC{$A, \varDelta \Longrightarrow B$}
     \RightLabel{$\to$導入}
     \UnaryInfC{$\varDelta \Longrightarrow A \to B$}
     \UnaryInfC{$\varGamma , \varDelta \Longrightarrow A \to B$}
     \LeftLabel{$\to$除去}
     \BinaryInfC{$\varGamma , \varDelta \Longrightarrow B$}
   \end{prooftree}
   よって,シークエント$\varGamma , \varDelta \Longrightarrow B$
   は導出可能である.
 \end{proof}

 cut規則は三段論法とも呼ばれている.
 実際,$\varGamma$をひとつの命題(条件)$P$であるとし,
 $A, B$をそれぞれ命題(条件)$Q, R$におきかえ,$\varDelta$を空列とみなせば,
 cut規則により
 \begin{prooftree}
   \AxiomC{$P \Longrightarrow Q$}
   \AxiomC{$Q \Longrightarrow R$}
   \BinaryInfC{$P \Longrightarrow R$}
 \end{prooftree}
 という推論ができることになる.
 今後はcut規則も推論法則として妥当なものとして利用していくことにする.

 次の定理\ref{thm:haityuritu}は,古典論理における非常に重要な定理である.
 \index[widx]{はいちゅうりつ@排中律 \, law of excluded middle}
 \begin{thm}[排中律] \label{thm:haityuritu}
   任意の命題(条件)$A$について,$A \lor \lnot A$は証明可能である.
 \end{thm}  
 \begin{proof}
   シークエント$\; \; \Longrightarrow A \lor \lnot A$を導出する. 
   \vspace{0.3cm} \\
   {\footnotesize
  % \begin{prooftree}
        \AxiomC{始式}
        \noLine
        \UnaryInfC{$A \Longrightarrow A$}
     \RightLabel{$\lor$導入}
        \UnaryInfC{$A \Longrightarrow A \lor \lnot A$}
        \UnaryInfC{$A ,  \lnot ( A \lor \lnot A)
                    \Longrightarrow A \lor \lnot A$}
        \AxiomC{始式}
        \noLine
        \UnaryInfC{$\lnot ( A \lor \lnot A ) \Longrightarrow \lnot ( A \lor \lnot A)$}
        \UnaryInfC{$A,  \lnot ( A \lor \lnot A ) 
                    \Longrightarrow \lnot ( A \lor \lnot A )$}
     \RightLabel{$\lnot$除去}
     \kernHyps{2.5cm}
        \BinaryInfC{$A , \lnot ( A \lor \lnot A ) 
                     \Longrightarrow \curlywedge$}
     \RightLabel{$\lnot$導入}
        \UnaryInfC{$\lnot ( A \lor \lnot A ) \Longrightarrow \lnot A $}
     \RightLabel{$\lor$導入}
        \UnaryInfC{$\lnot ( A \lor \lnot A ) \Longrightarrow A \lor \lnot A $}
        \AxiomC{始式}
        \noLine
        \UnaryInfC{$\lnot ( A \lor \lnot A ) 
                    \Longrightarrow \lnot ( A \lor \lnot A )$}
     \kernHyps{2cm}
     \RightLabel{$\lnot$除去}
     \insertBetweenHyps{\hskip -4.6cm}
        \BinaryInfC{$\lnot ( A \lor \lnot A ) \Longrightarrow \curlywedge$}
     \RightLabel{$\lnot$導入}
        \UnaryInfC{$\Longrightarrow \lnot \lnot ( A \lor \lnot A )$}
     \RightLabel{2重否定の除去}
        \UnaryInfC{$\Longrightarrow A \lor \lnot A$}
   %\end{prooftree}
   \DisplayProof
   } \vspace{0.3cm} \\ 
   従って,$A \lor \lnot A$は証明可能である.
 \end{proof}
 定理\ref{thm:haityuritu}の証明を言葉に翻訳してみよう.
 \begin{oframed}
   $\lnot ( A \lor \lnot A)$を仮定して矛盾を導く.
   もし$A$であるとすると,$A \lor \lnot A$となり,
   これは$\lnot ( A \lor \lnot A)$に矛盾する.
   従って,$\lnot A$でなければならない.
   しかしこの場合も$A \lor \lnot A$となり,
   やはり$\lnot ( A \lor \lnot A)$に矛盾する.
   よって,$\lnot \lnot ( A \lor \lnot A )$,
   すなわち$A \lor \lnot A$となる.
 \end{oframed}
 
 どの箇所でどの推論規則が使われているか確認してほしい.
 言葉での証明においては「仮定する」という文言が出てきたが,
 シークエント計算においては始式を持ち出したことに相当する.
 
 定理\ref{thm:haityuritu}の証明では,証明したい命題$A \lor \lnot A$の
 否定$\lnot ( A \lor \lnot A )$を仮定し,矛盾を導くことによって
 証明を行った.
 一般に,命題(条件)$A$を証明する代わりに$A$の否定$\lnot A$を
 仮定し,そこから矛盾を導く証明法のことを
 \index[widx]{はいりほう@背理法 \, reductio ad absurdum}
 \emph{背理法}(reductio ad absurdum)という.
 背理法においては本質的に2重否定の除去が使われていることに注意せよ
 \footnote{$A$を仮定して矛盾を導き,そこから$\lnot A$
 を結論する証明法も背理法と呼ばれることがあるが,
 この2つは厳密には区別して扱われるべき証明法である.}
 .
 
 \begin{que} \label{que:haityurituouyou}
   $\sqrt{2} ^{\sqrt{2}}$は明らかに実数である.
   命題$A$を「$\sqrt{2}^{\sqrt{2}}$は有理数である」と定めたとき,
   $A$に排中律を適用することで,
   無理数$a,  b$で$a^b$が有理数になるものが存在することを示せ.
 \end{que}

 定理\ref{thm:haityuritu}の証明では,推論規則の適用過程を図に書き表してある.
 一般に,シークエントへの推論規則の適用過程を表した図形を
 \index[widx]{どうしゅつず@導出図 \, derivation diagram}
 \emph{導出図}(derivation diagram)と呼ぶ.

 2重否定については少し補足をしておこう.
 \begin{lemma} \label{lemma:nijuhitei} 
   命題(条件)$A$について,
   \begin{align}
     \lnot \lnot A \equiv A
     \label{eq:nijuhitei}
   \end{align}
   が成り立つ.
 \end{lemma}
 \begin{proof}
   まずはシークエント$\lnot \lnot A \Longrightarrow A$を導出しよう.
   \begin{prooftree}
     \AxiomC{始式}
     \noLine
     \UnaryInfC{$\lnot \lnot A \Longrightarrow \lnot \lnot A$}
    \LeftLabel{2重否定の除去}
     \UnaryInfC{$\lnot \lnot A \Longrightarrow A$}
   \end{prooftree}
   従って,シークエント$\lnot \lnot A \Longrightarrow A$は導出可能である.

   次に,シークエント$A \Longrightarrow \lnot \lnot A$を導出する.
   \begin{prooftree}
     \AxiomC{始式}
     \noLine
     \UnaryInfC{$A \Longrightarrow A$}
     \UnaryInfC{$A ,  \lnot A \Longrightarrow A$}
     \AxiomC{始式}
     \noLine
     \UnaryInfC{$\lnot A \Longrightarrow \lnot A$}
     \UnaryInfC{$A ,  \lnot A \Longrightarrow \lnot A$}
    \LeftLabel{$\lnot$除去}
     \BinaryInfC{$A,  \lnot A \Longrightarrow \curlywedge$}
    \LeftLabel{$\lnot$導入}
     \UnaryInfC{$A \Longrightarrow \lnot \lnot A$}
   \end{prooftree}
   よって,シークエント$A \Longrightarrow \lnot \lnot A$も導出可能であり,
   従って式\eqref{eq:nijuhitei}が成り立つ.
 \end{proof}
 補題\ref{lemma:nijuhitei}の証明において,
 シークエント$A \Longrightarrow \lnot \lnot A$は2重否定の除去の
 推論規則を使わずに導出できたことに留意されたい.

 シークエントを導出する際,導出図を構成するのが基本となるが,
 シークエントに番号を振って書き並べる方法もある.
 次の定理\ref{thm:ganilor}の証明で実例を見せることにする.
 
 \begin{thm} \label{thm:ganilor}
   命題(条件)$A,  B$に対し,
   \begin{align}
     A \to B \equiv \lnot A \lor B
   \end{align}
   が成り立つ.
 \end{thm}
 \begin{proof} 
   まずは,シークエント$A \to B \Longrightarrow \lnot A \lor B$
   を導出する.
   \begin{enumerate}[1. ]
     \item $A \to B \Longrightarrow A \to B $ \quad [始式] 
     \item $\lnot ( \lnot A \lor B ) \Longrightarrow 
            \lnot ( \lnot A \lor B ) $ \quad [始式(背理法で示す)]
     \item $A \Longrightarrow A$ \quad [始式]
     \item $A ,  A \to B \Longrightarrow B $ \quad [$1.,3.$から$\to$除去による]
     \item $A ,  A \to B \Longrightarrow \lnot A \lor B$ 
            \quad [4.から$\lor$導入による]
     \item $A ,  A \to B ,  \lnot ( \lnot A \lor B ) 
            \Longrightarrow \curlywedge $ \quad [$2.,5.$から$\lnot$除去による] 
     \item $A \to B ,  \lnot ( \lnot A \lor B ) 
            \Longrightarrow \lnot A $ \quad [6.から$\lnot$導入による] 
     \item $A \to B ,  \lnot ( \lnot A \lor B ) 
            \Longrightarrow \lnot A \lor B 
            $ \quad [7.から$\lor$導入による] 
     \item $A \to B ,  \lnot ( \lnot A \lor B ) 
            \Longrightarrow \curlywedge $ \quad [$2.,8.$から$\lnot$除去による] 
     \item $A \to B \Longrightarrow \lnot \lnot ( \lnot A \lor B ) 
            $ \quad [9.から$\lnot$導入による] 
     \item $ A \to B \Longrightarrow \lnot A \lor B $ \quad [10.から2重否定の除去による]
   \end{enumerate}
   次に,シークエント$\lnot A \lor B \Longrightarrow A \to B$を導出する.
   \begin{enumerate}[1. ]
     \item $\lnot A \lor B \Longrightarrow \lnot A \lor B$ \quad [始式]
     \item $A \Longrightarrow A$ \quad [始式]
     \item $\lnot A \Longrightarrow \lnot A$ \quad [始式]
     \item $A,  \lnot A \Longrightarrow \curlywedge$ \quad [$2., 3.$から$\lnot$除去による]
     \item $A ,  \lnot A \Longrightarrow B$ \quad [4.から矛盾による]
     \item $B \Longrightarrow B$ \quad [始式]
     \item $A,  \lnot A \lor B \Longrightarrow B$ \quad [$1., 5., 6.$から$\lor$除去による]
     \item $\lnot A \lor B \Longrightarrow A \to B$ \quad [7.から$\to$導入による]
   \end{enumerate}
   従って,$A \to B \equiv \lnot A \lor B$が成り立つ.
   \end{proof}
   定理\ref{thm:ganilor}の証明のように,表記を簡単にするためにも,
   シークエントの導出において,前提の増加の適用は省略することにする.
   
   \index[widx]{De Morganのほうそく@De Morganの法則}
   \index[nidx]{De Morgan@De Morgan(ド・モルガン)}
   \begin{thm}[命題論理におけるDe Morganの法則] \label{thm:demorganmeidai}
     命題(条件)$A,  B$について,
     \begin{align}
       \lnot ( A \lor B ) \equiv \lnot A \land \lnot B
       \label{eq:demorgan1} \\
       \lnot ( A \land B ) \equiv \lnot A \lor \lnot B
       \label{eq:demorgan2}
     \end{align}
     が成り立つ.
   \end{thm}
   \begin{proof}
     式\eqref{eq:demorgan1}のみ示す.式\eqref{eq:demorgan2}も同様である.
     \begin{enumerate}[1. ]
       \item $\lnot ( A \lor B ) \Longrightarrow \lnot ( A \lor B ) $ \quad [始式]
       \item $ A \Longrightarrow A$ \quad [始式]
       \item $A \Longrightarrow A \lor B $ \quad [2.から$\lor$導入による]
       \item $A,  \lnot ( A \lor B ) \Longrightarrow \curlywedge$ 
              \quad [$1., 3.$から$\lnot$除去による]
       \item $\lnot ( A \lor B ) \Longrightarrow \lnot A $ \quad [4.から$\lnot$導入による]
       \item $B \Longrightarrow B$ \quad [始式]
       \item $B \Longrightarrow A \lor B$ \quad [6.から$\lor$導入による]
       \item $B,  \lnot ( A \lor B ) \Longrightarrow \curlywedge$
              \quad [$1., 7.$から$\lnot$除去による]
       \item $\lnot ( A \lor B ) \Longrightarrow \lnot B$ 
              \quad [8.から$\lnot$導入による]
       \item $\lnot ( A \lor B ) \Longrightarrow \lnot A \land \lnot B$
              \quad [$5., 9.$から$\land$導入による]
     \end{enumerate}
     よって,シークエント$\lnot ( A \lor B ) \Longrightarrow \lnot A \land \lnot B$
     は導出可能である.
     \begin{enumerate}[1. ]
       \item $\lnot A \land \lnot B \Longrightarrow \lnot A \land \lnot B$
              \quad [始式]
       \item $A \lor B \Longrightarrow A \lor B $ \quad [始式]
       \item $A \Longrightarrow A$ \quad [始式]
       \item $\lnot A \land \lnot B \Longrightarrow \lnot A $ \quad [1.から$\land$除去による]
       \item $A,  \lnot A \land \lnot B \Longrightarrow \curlywedge$
              \quad [$3., 4.$から$\lnot$除去による]
       \item $B \Longrightarrow B$ \quad [始式]
       \item $\lnot A \land \lnot B \Longrightarrow \lnot B$ \quad [1.から$\land$除去による]
       \item $B,  \lnot A \land B \Longrightarrow \curlywedge$ \quad [$6., 7.$から$\lnot$除去による]
       \item $A \lor B , \lnot A \land \lnot B \Longrightarrow \curlywedge$
              \quad [$2., 5., 8.$から$\lor$除去による]
       \item $\lnot A \land \lnot B \Longrightarrow \lnot ( A \lor B )$
              \quad [9.から$\lnot$導入による]
     \end{enumerate}
     よって,シークエント$\lnot A \land \lnot B \Longrightarrow \lnot ( A \lor B )$
     は導出可能である.

     以上より,$\lnot ( A \lor B ) \equiv \lnot A \land \lnot B$となる.
   \end{proof}

   述語論理についても,命題論理におけるDe Morganの法則
   と類似した関係が成り立つ.
   \index[widx]{De Morganのほうそく@De Morganの法則}
   \index[nidx]{De Morgan@De Morgan(ド・モルガン)}
   \begin{thm}[述語論理におけるDe Morganの法則] \label{thm:demorganjutugo}
     任意の述語$F$について
     \begin{align}
       \lnot \forall x F(x) \equiv \exists x \lnot F(x)
       \label{eq:demorgan3} \\
       \lnot \exists x F(x) \equiv \forall x \lnot F(x)
       \label{eq:demorgan4}
     \end{align}
     が成り立つ.
   \end{thm}
   \begin{proof}
     式\eqref{eq:demorgan3}の導出をする.
     まず$\lnot \forall x F(x) \Longrightarrow \exists x \lnot F(x)$を導く.
     \begin{enumerate}[1. ]
       \item $\lnot \forall x F(x) \Longrightarrow \lnot \forall x F(x)$
              \quad [始式]
       \item $\lnot \exists x \lnot F(x) \Longrightarrow \lnot \exists x \lnot F(x) $
              \quad [始式(背理法で示す)]
       \item $\lnot F(a) \Longrightarrow \lnot F(a)$ \quad [始式($a$は新たな自由変数)]
       \item $\lnot F(a) \Longrightarrow \exists x \lnot F(x)$ 
              \quad [3.から$\exists$導入による]
       \item $\lnot \exists x \lnot F(x) ,  \lnot F(a) 
              \Longrightarrow \curlywedge$ \quad [$2., 4.$から$\lnot$除去による]
       \item $\lnot \exists x \lnot F(x) \Longrightarrow \lnot \lnot F(a)$
              \quad [5.から$\lnot$導入による]
       \item $\lnot \exists x \lnot F(x) \Longrightarrow F(a)$
              \quad [6.から2重否定の除去による]
       \item $\lnot \exists x \lnot F(x) \Longrightarrow \forall x F(x)$
              \quad [7.から$\forall$導入による]
       \item $\lnot \exists x \lnot F(x) ,  \lnot \forall x F(x) 
              \Longrightarrow \curlywedge$ \quad [$1., 8.$から$\lnot$除去による]
       \item $\lnot \forall x F(x) \Longrightarrow \lnot \lnot \exists x \lnot F(x)$
              \quad [9.から$\lnot$導入による]
       \item $\lnot \forall x F(x) \Longrightarrow \exists x \lnot F(x)$
              \quad [10.から2重否定の除去による]
     \end{enumerate}
     \begin{oframed}
       言葉への翻訳:
       $\lnot \forall x F(x) $という仮定のもと,$\exists x \lnot F(x)$を導くことを考える.
       $\lnot \exists x \lnot F(x)$だとして矛盾を導くことにする.
       自由変数$a$を任意にとって$F(a)$が成り立つことを示したいのだが,
       これも$\lnot F(a)$を仮定して矛盾を導くことにする.
       
       さて,$\lnot F(a)$より$\exists \lnot x F(x)$であるが,
       これは$\lnot \exists x \lnot F(x)$に矛盾する.
       従って$\lnot \lnot F(a)$,すなわち$F(a)$である.
       自由変数$a$は任意だったので$\forall x F(x)$となり,
       これは$\lnot \forall x F(x)$に矛盾する.
       以上より,$\lnot \lnot \exists x \lnot F(x) $,すなわち$\exists x \lnot F(x)$となる.
     \end{oframed}

     次に,$\exists x \lnot F(x) \Longrightarrow \lnot \forall x F(x) $を導く.
     \begin{enumerate}[1. ]
       \item $\exists x \lnot F(x) \Longrightarrow \exists x \lnot F(x)$
              \quad [始式]
       \item $\forall x F(x) \Longrightarrow \forall x F(x) $
              \quad [始式]
       \item $\lnot F(a) \Longrightarrow \lnot F(a) $ \quad [始式($a$は新たな自由変数)]
       \item $\forall x F(x) \Longrightarrow F(a)$ \quad [2.から$\forall$除去による]
       \item $\forall x F(x) ,  \lnot F(a) \Longrightarrow \curlywedge$
              \quad [$3., 4.$から$\lnot$除去による]
       \item $\exists x \lnot F(x) ,  \forall x F(x) \Longrightarrow \curlywedge$
              \quad [$1., 5.$から$\exists$除去による]
       \item $\exists x \lnot F(x) \Longrightarrow \lnot \forall F(x)$
              \quad [6.から$\lnot$除去による]
     \end{enumerate}
     \begin{oframed}
       言葉への翻訳:$\exists x \lnot F(x)$という仮定のもと,$\lnot \forall x F(x)$
       を導くことを考える.$\forall x F(x)$を仮定して矛盾を導くことにする.

       $\exists x \lnot F(x)$だから,$\lnot F(a)$となる自由変数$a$が存在する.
       しかし,$\forall x F(x)$だから,$F(a)$とならなくてはならず矛盾する.
       これは$a$のとり方に依存しないので,$\lnot \forall x F(x)$が導かれる.
     \end{oframed}
     従って,$\lnot \forall x F(x) \equiv \exists x \lnot F(x)$が成り立つ.
     式\eqref{eq:demorgan4}も同様である.
   \end{proof}
   $\land$と$\lor$の組み合わせについて考えよう.
   \index[widx]{ぶんぱいりつ@分配律 \, distributive law}
   \begin{thm}[命題論理における分配律] \label{thm:bunpaimeidai}
     命題(条件)$A,  B,  C$に対し,
     \begin{align}
       A \land ( B \lor C ) \equiv (A \land B ) \lor (A \land C) 
       \label{eq:bunpailand} \\
       A \lor ( B \land C) \equiv (A \lor B) \land ( A \lor C )
       \label{eq:bunpailor}
     \end{align}
     が成り立つ.
   \end{thm}
   \begin{proof}
     式\eqref{eq:bunpailand}を示す.式\eqref{eq:bunpailor}も同様に示せる.
     まずはシークエント
     $A \land (B \lor C) \Longrightarrow (A \land B) \lor (A \land C)$
     を導出しよう.
     \begin{enumerate}[1. ]
       \item $A \land ( B \lor C ) \Longrightarrow A \land ( B \lor C) $
              \quad [始式]
       \item $A \land ( B \lor C ) \Longrightarrow A$ \quad [1.から$\land$除去による]
       \item $A \land ( B \lor C) \Longrightarrow B \lor C $ \quad [1.から$\land$除去による]
       \item $B \Longrightarrow B$ \quad [始式]
       \item $B,  A \land ( B \lor C ) \Longrightarrow A \land B$ 
              \quad [$2., 4.$から$\land$導入による]
       \item $B,  A \land ( B \lor C) \Longrightarrow ( A \land B) \lor ( A \land C)$
              \quad [5.から$\lor$導入による]
       \item $C \Longrightarrow C$ \quad [始式]
       \item $C,  A \land ( B \lor C ) \Longrightarrow A \land C$
              \quad [$2., 7.$から$\land$導入による]
       \item $C,  A \land ( B \lor C ) \Longrightarrow ( A \land B ) \lor (A \land C)$
              \quad [8.から$\lor$導入による]
       \item $A \land ( B \lor C) \Longrightarrow (A \land B) \lor (A \land C)$
              \quad [$3., 6., 9.$から$\lor$除去による]
     \end{enumerate} 
     次に,シークエント$( A \land B ) \lor ( A \land C) \Longrightarrow A \land ( B \lor C)$
     を導出する.
     \begin{enumerate}[1. ]
       \item $(A \land B) \lor ( A \land C) \Longrightarrow (A \land B) \lor (A \land C)$
              \quad [始式]
       \item $A \land B \Longrightarrow A \land B$ \quad [始式]
       \item $A \land B \Longrightarrow A $ \quad [2. から$\land$除去による]
       \item $A \land B \Longrightarrow B$ \quad [2.から$\land$除去による]
       \item $A \land B \Longrightarrow B \lor C$ \quad [4.から$\lor$導入による]
       \item $A \land B \Longrightarrow A \land ( B \lor C)$
              \quad [$3., 5.$から$\land$導入による]
       \item $A \land C \Longrightarrow A \land C$ \quad [始式]
       \item $A \land C \Longrightarrow A$ \quad [7.から$\land$除去による]
       \item $A \land C \Longrightarrow C$ \quad [7.から$\land$除去による]
       \item $A \land C \Longrightarrow B \lor C$ \quad [9.から$\lor$導入による]
       \item $A \land C \Longrightarrow A \land ( B \lor C )$
              \quad [$8., 10.$から$\land$導入による]
       \item $(A \land B) \lor ( A \land C) \Longrightarrow A \land (B \lor C)$
              \quad [$1., 6., 11.$から$\lor$除去による]
     \end{enumerate}
     従って,$A \land (B \lor C) \equiv (A \land B) \lor (A \land C)$
     が成り立つ.
   \end{proof}
   \begin{que} \label{que:meidaiketugouritu}
     命題(条件)$A,  B,  C$について
     \begin{align}
       A \land B & \equiv B \land A
       \label{eq:landkoukan} \\
       A \lor B & \equiv B \lor A 
       \label{eq:lorkoukan} \\
       A \land ( B \land C) & \equiv (A \land B ) \land C 
       \label{eq:landketugouritu} \\
       A \lor ( B \lor C) & \equiv ( A \lor B ) \lor C
       \label{eq:lorketugouritu}
     \end{align}
     が成り立つことを示せ.
     このことから,$A \land B \land C$や$A \lor B \lor C$
     などと書くことが許されることがわかる.
   \end{que}
   定理\ref{thm:ganilor}と定理\ref{thm:demorganmeidai},
   定理\ref{thm:demorganjutugo}により,
   次の定理\ref{thm:ganisingi}が導かれる.
   \begin{thm} \label{thm:ganisingi}
     命題(条件)$A,  B$,および述語$F,  G$について,
     \begin{align}
       \lnot ( A \to B ) & \equiv A \land \lnot B
       \label{eq:ganisingi} \\
       \lnot \forall x ( F(x) \to G(x) ) 
       & \equiv \exists x ( F(x) \land \lnot G(x) )
       \label{eq:ganisingijutugo}
     \end{align}
     が成り立つ.
   \end{thm}
   \begin{proof}
     \begin{align*}
       \lnot ( A \to B ) & \equiv \lnot ( \lnot A \lor B) \\
       & \equiv \lnot \lnot A \land \lnot B \\
       & \equiv A \land \lnot B 
     \end{align*}
     これで式\eqref{eq:ganisingi}が導かれた.
     \begin{align*}
       \lnot \forall x ( F(x) \to G(x) ) 
       & \equiv \exists x \lnot ( F(x) \to G(x) ) \\
       & \equiv \exists x ( F(x) \land \lnot G(x) )
     \end{align*}
     よって,式\eqref{eq:ganisingijutugo}も導かれた.
   \end{proof}
   定理\ref{thm:ganisingi}の証明では証明済みの定理を用いたが,
   推論規則を利用して
   シークエントを導出することによっても(少し長くはなるが)証明できる.

   また,定理\ref{thm:ganisingi}の証明では,
   以下に挙げる定理\ref{thm:tikanteiri}を暗に用いている.
   証明はしないがその主張だけは提示しておく.
   \index[widx]{ちかんていり@置換定理}
   \begin{thm}[置換定理] \label{thm:tikanteiri}
     $\mathscr{F}(X)$を命題変数$X$を含む論理式とする.
     このとき,命題$A,B$について,$A \equiv B$ならば
     $\mathscr{F}(A) \equiv \mathscr{F}(B)$
     となる.
     また,$\mathscr{F}(P)$を述語変数$P$を含む論理式とする.
     述語$F,G$に対し,$\forall x (F(x) \rightleftarrows G(x))$
     が証明可能であるならば$\mathscr{F}(F) \equiv \mathscr{F}(G)$
     が成り立つ.
   \end{thm}
   ここで,「命題(述語)変数$X$を含む論理式」というのは,
   とりあえず命題(述語)$X$の真偽によってその真偽が決定される命題
   とでも思っておけばよい.たとえば,
   $A \land B$は命題$A,B$を含む論理式であり,
   $\forall x (F(x) \lor G(x))$は述語$F,G$を含む論理式である.

   定理\ref{thm:ganisingi}の証明では,
   $A \to B \equiv \lnot A \lor B$を根拠に
   $\lnot ( A \to B ) \equiv \lnot ( \lnot A \lor B)$を導き,
   $\lnot \lnot A \equiv A$を根拠に$\lnot \lnot A \land \lnot B \equiv A \land \lnot B$
   を導いた.また,任意の自由変数$a$に対して
   \begin{align*}
     \lnot ( F(a) \to G(a) ) \equiv F(a) \land \lnot G(a) 
   \end{align*}
   が成り立つ,すなわち
   \begin{align*}
     \forall x ( \lnot ( F(x) \to G(x) ) \rightleftarrows F(x) \land \lnot G(x) )
   \end{align*}
   が証明可能であることを根拠に
   \begin{align*}
     \exists x \lnot ( F(x) \to G(x) ) \equiv \exists x ( F(x) \land \lnot G(x) )
   \end{align*}
   を導いたが,ここに定理\ref{thm:tikanteiri}が使われている.

   \begin{que} \label{que:Dnarabanot}
     空でない集合$D$と$D$(を含んだ集合)を対象領域とする
     $x$の条件$F(x)$に対し,次式が成り立つことを示せ.
     \begin{align}
       \lnot \forall x \in D ( F(x) ) & \equiv \exists x \in D ( \lnot F(x) ) 
       \label{eq:Dnotall} \\
       \lnot \exists x \in D (F(x) ) & \equiv \forall x \in D ( \lnot F(x) )
       \label{eq:Dnotexists} 
     \end{align}
   \end{que}



   命題(条件)$A ,  B$について,
   $A \to B$の形をした命題(条件)について考える.
   命題(条件)$B \to A$を$A \to B$の
   \index[widx]{めいだい@命題 \, proposition!のぎゃく@---の逆 \, converse ---}
   \emph{逆}(converse)といい,
   命題(条件)$\lnot A \to \lnot B$を$A \to B$の
   \index[widx]{めいだい@命題 \, proposition!のうら@---の裏 \, inverse ---}
   \emph{裏}(inverse)という.
   また,命題(条件)$\lnot B \to  \lnot A$を
   $A \to B $の
   \index[widx]{めいだい@命題 \, proposition!のたいぐう@---の対偶 \, contraposition}
   \emph{対偶}(contraposition)と呼ぶ.
   \begin{thm} \label{thm:taiguu}
     命題(条件)$A,  B$について,
     \begin{align}
       A \to B \equiv \lnot B \to \lnot A 
       \label{eq:taiguudouti} \\
       B \to A \equiv \lnot A \to \lnot B
       \label{eq:gyakudouti}
     \end{align}
     が成り立つ.
   \end{thm}
   
   \begin{proof}
     式\eqref{eq:taiguudouti}を示す.
     シークエント$A \to B \Longrightarrow \lnot B \to \lnot A$
     を導出する.
     \begin{enumerate}[1. ]
       \item $A \to B \Longrightarrow A \to B$ \quad [始式]
       \item $\lnot B \Longrightarrow \lnot B$ \quad [始式]
       \item $A \Longrightarrow A $ \quad [始式]
       \item $A,  A \to B \Longrightarrow B$ \quad [$1., 3.$から$\to$除去による]
       \item $A,  A \to B ,  \lnot B \Longrightarrow \curlywedge$
              \quad [$2., 4.$から$\lnot$除去による]
       \item $A \to B ,  \lnot B \Longrightarrow \lnot A$ \quad [5.から$\lnot$導入による]
       \item $A \to B \Longrightarrow \lnot B \to \lnot A$ \quad [6.から$\to$導入による]
     \end{enumerate}
     次に,シークエント$\lnot B \to \lnot A \Longrightarrow A \to B$
     を導出する.
     \begin{enumerate}[1. ]
       \item $\lnot B \to \lnot A \Longrightarrow \lnot B \to \lnot A$
              \quad [始式]
       \item $A \Longrightarrow A $ \quad [始式]
       \item $\lnot B \Longrightarrow \lnot B$ \quad [始式]
       \item $\lnot B ,  \lnot B \to \lnot A \Longrightarrow \lnot A$
              \quad [$1., 3.$から$\to$除去による]
       \item $A,  \lnot B ,  \lnot B \to \lnot A \Longrightarrow \curlywedge$
              \quad [$2., 4.$から$\lnot$除去による]
       \item $A ,  \lnot B \to \lnot A \Longrightarrow \lnot \lnot B $
              \quad [5.から$\lnot$導入による]
       \item $A ,  \lnot B \to \lnot A \Longrightarrow B $
              \quad [6.から2重否定の除去による]
       \item $\lnot B \to \lnot A \Longrightarrow A \to B$
              \quad [7.から$\to$導入による]
     \end{enumerate}
     従って,式\eqref{eq:taiguudouti}は成り立つ.
     式\eqref{eq:gyakudouti}も同様にできる.
   \end{proof}
   \begin{que} \label{que:tototo}
     命題(条件)$A,  B,  C$について,
     \begin{align}
       A \to ( B \to C) \equiv A \land B \to C
       \label{eq:tototo}
     \end{align}
     が成り立つことを示せ.
   \end{que}

   \begin{que} \label{que:Peirce}
     命題(条件)$A,  B$について,シークエント
     \begin{align}
       (A \to B) \to A \Longrightarrow A
       \label{eq:Peircelaw}
     \end{align}
     を導出せよ.
     これを
     \index[widx]{Peirceのほうそく@Peirceの法則}
     \index[nidx]{Peirce@Peirce(パース)}
     \textbf{Peirceの法則}という.
   \end{que}

   限定記号と$\to$の組み合わせには注意が必要である.
   \begin{thm} \label{thm:genteito}
     述語$F$と$x$を含まない命題(条件)$A$に対し,
     \begin{align}
       \forall x ( F(x) \to A ) \equiv \exists x F(x) \to A 
       \label{eq:foralltoA} \\
       \exists x ( F(x) \to A ) \equiv \forall x F(x) \to A
       \label{eq:existstoA}
     \end{align}
     が成り立つ.
   \end{thm}
   \begin{proof}
     式\eqref{eq:foralltoA}の導出は容易であるから,式\eqref{eq:existstoA}
     の導出をする.
     シークエント$\exists x (F(x) \to A ) \Longrightarrow \forall x F(x) \to A$
     を導出しよう.
     \begin{enumerate}[1. ]
       \item $\exists x (F(x) \to A ) \Longrightarrow \exists x (F(x) \to A)$
              \quad [始式]
       \item $\forall x F(x) \Longrightarrow \forall x F(x)$ \quad [始式]
       \item $F(a) \to A \Longrightarrow F(a) \to A $ \quad [始式($a$)は新たな自由変数]
       \item $\forall x F(x) \Longrightarrow F(a)$ \quad [2.から$\forall$除去による]
       \item $\forall x F(x) ,  F(a) \to A \Longrightarrow A$
              \quad [$3., 4.$から$\to$除去による]
       \item $\exists x (F(x) \to A) ,  \forall x F(x) \Longrightarrow A$
              \quad [$1., 5.$から$\exists$除去による]
       \item $\exists x (F(x) \to A) \Longrightarrow \forall x F(x) \to A$
              \quad [6.から$\to$導入による]
     \end{enumerate}
     次に,シークエント$\forall x F(x) \to A \Longrightarrow \exists x (F(x) \to A)$
     を導出する.
     \begin{enumerate}[1. ]
       \item $\forall x F(x) \to A \Longrightarrow \forall x F(x) \to A $
              \quad [始式]
       \item $\lnot \exists x (F(x) \to A) \Longrightarrow \lnot \exists x (F(x) \to A)$
              \quad [始式(背理法で示す)]
       \item $\lnot F(a) \Longrightarrow \lnot F(a)$ \quad [始式($a$は新たな自由変数)]
       \item $F(a) \Longrightarrow F(a)$ \quad [始式]
       \item $\lnot F(a) ,  F(a) \Longrightarrow \curlywedge$
              \quad [$3., 4.$から$\lnot$除去による]
       \item $\lnot F(a) ,  F(a) \Longrightarrow A$ \quad [5.から矛盾による]
       \item $\lnot F(a) \Longrightarrow F(a) \to A$ \quad [6.から$\to$導入による]
       \item $\lnot F(a) \Longrightarrow \exists x (F(x) \to A)$
              \quad [7.から$\exists$導入による]
       \item $\lnot F(a) ,  \lnot \exists x (F(x) \to A) \Longrightarrow \curlywedge$
              \quad [$2., 8.$から$\lnot$除去による]
       \item $\lnot \exists x (F(x) \to A) \Longrightarrow \lnot \lnot F(a)$
              \quad [9.から$\lnot$導入による]
       \item $\lnot \exists x (F(x) \to A) \Longrightarrow F(a)$ 
              \quad [10.から2重否定の除去による]
       \item $\lnot \exists x (F(x) \to A) \Longrightarrow \forall x F(x) $
              \quad [11.から$\forall$導入による]
       \item $\forall x F(x) \to A ,  \lnot \exists x (F(x) \to A) \Longrightarrow A$
              \quad [$1., 12.$から$\to$除去による]
       \item $F(a) ,  \forall x F(x) \to A ,  \lnot \exists x (F(x) \to A)
              \Longrightarrow A $ \quad [13.から前提の増加による]
       \item $\forall x F(x) \to A ,  \lnot \exists x(F(x) \to A)
              \Longrightarrow F(a) \to A$ \quad [14.から$\to$導入による]
       \item $\forall x F(x) \to A ,  \lnot \exists x(F(x) \to A)
              \Longrightarrow \exists x(F(x) \to A)$ \quad [15.から$\exists$導入による]
       \item $\forall x F(x) \to A ,  \lnot \exists x (F(x) \to A)
              \Longrightarrow \curlywedge$ \quad [$2., 16.$から$\lnot$除去による]
       \item $\forall x F(x) \to A \Longrightarrow \lnot \lnot \exists x (F(x) \to A)$
              \quad [17.から$\lnot$導入による]
       \item $\forall x F(x) \to A \Longrightarrow \exists x (F(x) \to A) $
              \quad [18.から2重否定の除去による]
     \end{enumerate}
     以上で式\eqref{eq:existstoA}は示された.
   \end{proof}
   \begin{que} \label{que:togentei}
     述語$F$と$x$を含まない命題(条件)$A$に対し
     \begin{align}
       \forall x (A \to F(x) ) & \equiv A \to \forall x F(x) 
       \label{eq:Atoforall} \\
       \exists x (A \to F(x) ) & \equiv A \to \exists x F(x)
       \label{eq:Atoexists}
     \end{align}
     が成り立つことを示せ.
   \end{que}
   最後に限定記号と$\land,  \lor$の組み合わせについて言及しておこう.
   \begin{thm} \label{thm:genteilandor}
     述語$F,  G$に対して
     \begin{align}
       \forall x (F(x) \land G(x)) \equiv \forall x F(x) \land \forall x G(x)
       \label{eq:forallland} \\
       \exists x (F(x) \lor G(x)) \equiv \exists xF(x) \lor \exists xG(x)
       \label{eq:existslor}
     \end{align}
     が成り立つ.
   \end{thm}
   \begin{proof}
     容易である.
   \end{proof}
   \begin{que} \label{que:genteilandor}
     述語$F,  G$と$x$を含まない命題(条件)$A$について
     \begin{align}
       \forall x (F(x) \lor A) \equiv \forall x F(x) \lor A
       \label{eq:foralllorA} \\
       \exists x (F(x) \land A) \equiv \exists x F(x) \land A
       \label{eq:existslandA}
     \end{align}
     が成り立つことを示せ.
   \end{que}
   \begin{que} \label{que:genteikoukan}
     2変数の述語$F$に対し,
     \begin{align}
       \forall x \forall y F(x,y) \equiv \forall y \forall x F(x,y)
       \label{eq:forallkoukan} \\
       \exists x \exists y F(x,y) \equiv \exists y \exists x F(x,y)
       \label{eq:existskoukan}
     \end{align}
     が成り立つことを示せ.
   \end{que}
   \begin{que} \label{que:circdouti}
     $n$個の命題(条件)$P_1,  P_2,  \ldots ,  P_n$について,
     シークエント
     \begin{align*}
       P_1 \Longrightarrow P_2 , \  P_2 \Longrightarrow P_3  , \  
       \ldots  ,\   P_{n-1} \Longrightarrow P_n  ,\   P_n \Longrightarrow P_1
     \end{align*}
   \end{que}
   がすべて導出可能であるとする.このとき,
   $P_1,  P_2,  \ldots ,  P_n$はすべて同値であることを示せ.
     %
     
   問\ref{que:circdouti}の結果は,複数の命題がすべて同値であることを示すときに
   その手順を簡略化する強力な手段となる.
   たとえば,3つの命題$A,  B,  C$がすべて同値であることを示すのには,
   3つのシークエント
   \begin{align*}
     A \Longrightarrow B  ,\   B \Longrightarrow C  ,\   C \Longrightarrow A
   \end{align*}
   を導くだけでよいのである.  
   %
     %
 \section{等号について}
 \label{sec:equal}
   %
   %
   %
   %
   何か2つのものが等しいというとき,我々は「$=$」という記号を使ってきた.
   この「$=$」という記号は,
   数や関数,行列や図形などの様々な対象に対して使われる.
   個々の分野においては,
   議論の対象になるものに応じて「$=$」の使い方が決定される.
   たとえば,数$a,  b$に対して$a = b$が成り立つことと,
   図形$S,  T$に対して
   $S=T$が成り立つことの定義は異なるものである.
   しかしながら,同じ記号を使うだけあって,この「$=$」が示す「意味」
   はほとんど同じものである.
   この節では,「$=$」という記号を一段上の視点から眺めてみよう.

   \paragraph{等号公理}
   基礎となる公理をもとに等号について議論しよう.
   この資料で採用する公理は次の2つである.
   
   \index[widx]{とうごう@等号 \, equal sign}
   \begin{axiom}[等号公理]
     等号「$=$」は2つの項の関係を表す記号であって,
     \begin{enumerate}[1. ]
       \item 任意の項$s$に対して$s =s$が成り立つ.
       \item 任意の述語$F$,および任意の項$s,  t$に対して,
             シークエント
             $s=t \Longrightarrow F(s) \to F(t) $
             は導出可能である.
     \end{enumerate}
     を満たすものであるとする.
   \end{axiom}
   性質1を\index[widx]{はんしゃりつ@反射律 \, reflexive law}
   \emph{反射律}(reflexive law),
   性質2を
   \index[widx]{ちかんほうそく@置換法則}
   \emph{置換法則} という.

   今導入した等号公理が妥当であるかどうかは,
   この公理から導かれる定理を考察するよりほかはない.
   \index[widx]{だいにゅうげんり@代入原理}
   \begin{thm}[代入原理] \label{thm:dainyugenri}
     任意の関数記号$f$と任意の項$a,  b$に対して,シークエント
     \begin{align}
       a=b \Longrightarrow  f(a)=f(b)
       \label{eq:dainyu}
     \end{align}
     は導出可能である.
   \end{thm}
   \begin{proof}
     任意の項$a$に対し,$f(a)=f(x)$を$x$に関する述語と考え,
     これを$F(x)$とおくと,シークエント
     \begin{align*}
       a = b \Longrightarrow f(a)=f(a) \to f(a)=f(b)
     \end{align*}
     は
     \begin{align*}
       a=b \Longrightarrow F(a) \to F(b)
     \end{align*}
     と書き換えられ,置換法則の特別な場合と考えられることに注意する.
     \begin{enumerate}[1. ]
       \item $a=b \Longrightarrow f(a)=f(a) \to f(a) = f(b)$ \quad [置換法則]
       \item $\qquad \Longrightarrow f(a)=f(a)$ \quad [反射律]
       \item $a=b \Longrightarrow f(a) = f(b) $ \quad [$1., 2.$から$\to$除去による]
     \end{enumerate}
     従って,シークエント$a=b \Longrightarrow f(a) = f(b)$
     は導出可能である.
   \end{proof}
   定理\ref{thm:dainyugenri}において,「関数記号」という用語が出てきたが,
   これはとりあえず「項に依存して別の項をつくる記号」
   とでも思っておけばよい.

   等号「$=$」は次の2つの性質を満たす.
   証明は演習問題としよう.
   \begin{thm} \label{thm:taisyousuii}
     任意の項$a,  b,  c$に対して,次の2つのシークエント
     \begin{align}
       a = b \Longrightarrow b = a 
       \label{eq:taisyouritu} \\
       a =b \land b=c \Longrightarrow a =c
       \label{eq:suiiritu}
     \end{align}
     はともに導出可能である.
     \end{thm}
     式\eqref{eq:taisyouritu}を
     \index[widx]{たいしょうりつ@対象律 \, symmetric law}
     \emph{対称律}(symmetric law),
     式\eqref{eq:suiiritu}を
     \index[widx]{すいいりつ@推移律 \, transitive law}
     \emph{推移律}(transitive law)という.
     \begin{que} \label{que:taisyousuii}
       定理\ref{thm:taisyousuii}を証明せよ.
     \end{que}

     対称律を用いると,次の定理\ref{thm:equivtaisyou}が成り立つことがわかる.
     \begin{thm} \label{thm:equivtaisyou}
       任意の述語$F$,および任意の項$s,  t$に対し,シークエント
       \begin{align}
         s = t \Longrightarrow F(s) \rightleftarrows F(t)
         \label{eq:equivtaisyou}
       \end{align}
       は導出可能である.
     \end{thm}

    \paragraph{唯一存在記号}
     等号を用いると,述語$F$に対して「$F(x)$となる$x$がただひとつ存在する」
     という命題を表現することができる.
     たとえば,
     \begin{align}
       \exists x ( F(x) \land \forall y (F(y) \to y=x))
       \label{eq:tadahitotu}
     \end{align}
     という命題が「$F(x)$をとなる$x$がただ1つ存在する」という命題を表すことになる.
     この命題を
     \begin{align}
       \exists ! x F(x)
       \label{eq:existsbikkuri}
     \end{align}
     と表すことがある\footnote{現代ではあまり使われない記号である.}.

     項$s,  t$に対し,条件$\lnot(s=t)$を$s \neq t$と略記する.
     記号$\neq$を用いると,述語$F$に対し,
     「$F(x)$となる$x$は少なくとも2つ存在する」
     という命題が
     \begin{align}
       \exists x, y ((F(x) \land F(y) ) \land x \neq y)
       \label{eq:2tusonzai}
     \end{align}
     と表現できる.
     \begin{que} \label{que:2tusonzaihitei}
       述語$F$に対し,
       \begin{align*}
         \lnot \exists x,y ((F(x) \land F(y)) \land x \neq y)
         \equiv \forall x,y (F(x) \land F(y)  \to x=y)
       \end{align*}
       が成り立つことを示せ.
     \end{que}
     \begin{que} \label{que:tadahitotuexists}
       述語$F$に対し,シークエント
       \begin{align}
         \exists ! x F(x)
         \Longrightarrow \exists x F(x)
         \label{eq:existsunique} \\
         \exists ! x F(x)
         \Longrightarrow \exists x \forall y ( F(y) \to y=x )
         \label{eq:existstenceunique} \\
         \exists x \forall y (F(y) \to y=x ) \Longrightarrow 
         \forall x, y (F(x) \land F(y) \to x=y)
         \label{eq:uniquetakadaka} \\
         \exists x F(x) \land \forall x,y (F(x) \land F(y) \to x=y) 
         \Longrightarrow \exists ! x F(x)
         \label{eq:takadakaunique} \\
         \lnot \exists x F(x) \Longrightarrow \exists x \forall y (F(y) \to y=x)
         \label{eq:lnottakadaka}
       \end{align}
       を導出せよ.ただし,$\exists ! x F(x)$は$\exists x (F(x) \land \forall y (F(y) \to y=x))$
       の略記であるとする.
     \end{que}

     問\ref{que:tadahitotuexists}により,
     \begin{align}
       \exists ! x F(x) \equiv \exists x F(x) \land \forall x, y (F(x)\land F(y) \to x=y)
       \label{eq:uniquedouti} \\
       \exists x \forall y (F(y) \to y=x) \equiv \forall x, y (F(x) \land F(y) \to x=y)
       \label{eq:takadakadouti}
     \end{align}
     が成り立つことがわかる(式\eqref{eq:takadakadouti}に関しては
     排中律を思い出せ).
     式\eqref{eq:uniquedouti}は「$F(x)$となる$x$がただ1つ存在する」
     という命題の別の表記を与えており,
     式\eqref{eq:takadakadouti}は「$F(x)$となる$x$はたかだか1つである」
     という命題の2つの表記法を与えている.
 % 記号論理
 % \input{inputyou/set/set} % 集合
 % \input{inputyou/realnumber/realnumber} % 実数
 % \input{inputyou/cardinal/cardinal} % 濃度
%
%%%%%=========付録===========%%%%%%%%%
%
%
%
\appendix % 付録
%%%%%%%%%%============付録用に===========%%%%%%%%%
%
\makeatletter
 \renewcommand{\theequation}
  {\Alph{chapter}.\arabic{section}.\arabic{equation}}
   \@addtoreset{equation}{section} % 数式番号
\makeatother
 \renewcommand{\thesection}{\S \  \Alph{chapter}.\arabic{section}}
 \renewcommand{\thefigure}{\Alph{chapter}.\arabic{section}.\arabic{figure}} % 図の番号
 \renewcommand{\thetable}{\Alph{chapter}.\arabic{section}.\arabic{table}} % 表の番号
%
%
%
% \input{inputyou/number/number} % 数の体系
%
%
%
%%%%%%%===========あとがき==============%%%%%%%%
\backmatter
{\small
\chapter{演習問題略解} \label{answer}
\begin{description}
  \item[\refque{chp:sequent.sec:ronri.que:singihantei}] \mbox{}
  \begin{enumerate}
   \item 偽
   \item 真
   \item 真
   \item 偽(反例は$x=3,  y=2$など)
   \item 偽
   \item 真
  \end{enumerate}
\item[\refque{chp:sequent.sec:hensuu.que:sokubakusingi}] \mbox{} 
  \begin{enumerate}
   \item 偽(反例は$x=1$など)
   \item 真
   \item 偽
   \item 真
   \item 真 ($x=3$とでもすればよい)
   \item 偽
  \end{enumerate}
\item[\refque{chp:sequent.sec:hituyoujubun.que:xhituyoujubun}] \mbox{} 
  \begin{enumerate}
   \item 正しくない
   \item 正しい
   \item 正しい
  \end{enumerate}
\item[\refque{que:kigoukaranihongo}] \mbox{} \\
  ここに挙げるのはあくまで一例である.
  細かい言い回しを考えれば,解答として適切なものはいくつも考えられるであろう.
  \begin{enumerate}
    \item  $F(x)$となる$x$がただひとつ存在する.
    \item 任意の$\varepsilon >0$に対して$\delta >0$が存在して,
      $0< \lvert x- a \rvert < \delta$を満たす任意の$x \in I$に対して
      $\lvert f(x) -A \rvert < \varepsilon$が成り立つ.
    \item 任意の$y \in Y$に対して$x \in X$が存在して,$y=f(x)$となる.
    \item $f(x_1 )=f(x_2) $を満たす任意の$x_1, x_2 \in X$に対して$x_1 = x_2$が成り立つ.
    \item 任意の$a,b >0$に対して自然数$N$で$Na >b$となるものがとれる.
  \end{enumerate}

\item[\refque{que:nihongokarakigou}] \mbox{} \\
  \begin{enumerate}
    \item $\forall x ( x \in A \to x \in B).$
      (「必ず」とあることを考えれば$x$には全称記号をつけるのが妥当であろう)
    \item $\forall \varepsilon >0 \exists N \in \mathbb{N} \forall n,m \in \mathbb{N}
      (n,m \geq N \to \lvert a_n - a_m \rvert < \varepsilon ).$
    \item $\forall x,y \in \mathbb{R} (x^2+y^2=1 \to \exists \theta \in \mathbb{R} 
      ( 0 \leq \theta <2 \pi \land (x= \cos \theta \land y= \sin \theta ))).$
    \item $\exists c \in \mathbb{R}(\forall x \in S ( x \leq c ) \land 
      \forall M \in \mathbb{R} (\forall x \in S (x \leq M) \to c \leq M)).$
  \end{enumerate}
\item[\refque{que:haityurituouyou}] \mbox{} \\  
  $\sqrt{2}^{\sqrt{2}}$が有理数の場合には$a = b= \sqrt{2}$とすればよい.
  $\sqrt{2}^{\sqrt{2}}$が有理数でない場合には,
  $\sqrt{2}^{\sqrt{2}}$は無理数であるから
  $a = \sqrt{2} ^{\sqrt{2}},  b= \sqrt{2}$とすればよい.

\item[\refque{que:meidaiketugouritu}] \mbox{} \\
  式\eqref{eq:lorketugouritu}の証明のうち,
  シークエント$A \lor ( B \lor C ) \Longrightarrow (A \lor B) \lor C$
  の導出をする(ほかのものは省略する).
  \begin{enumerate}[1. ]
    \item $A \lor ( B \lor C) \Longrightarrow A \lor (B \lor C)$
           \quad [始式]
    \item $A \Longrightarrow A$ \quad [始式]
    \item $A \Longrightarrow A \lor B$ \quad [2.から$\lor$導入による]
    \item $A \Longrightarrow (A \lor B ) \lor C$ \quad [3.から$\lor$導入による]
    \item $B \lor C \Longrightarrow B \lor C$ \quad [始式]
    \item $B \Longrightarrow B$ \quad [始式]
    \item $B \Longrightarrow A \lor B$ \quad [6.から$\lor$導入による]
    \item $B \Longrightarrow ( A \lor B) \lor C$ \quad [7.から$\lor$導入による]
    \item $C \Longrightarrow C$ \quad [始式]
    \item $C \Longrightarrow (A \lor B ) \lor C$ \quad [9.から$\lor$導入による]
    \item $B \lor C \Longrightarrow (A \lor B ) \lor C$
           \quad [$5., 8., 10.$から$\lor$除去による]
    \item $A \lor ( B \lor C) \Longrightarrow (A \lor B ) \lor C$
           \quad [$1., 4., 11.$から$\lor$除去による]
  \end{enumerate}

\item[\refque{que:Dnarabanot}] \mbox{} \\
  式\eqref{eq:Dnotall}を示す.
  \begin{align*}
    \lnot \forall x \in D ( F(x) ) & \equiv \lnot \forall x ( x \in D \to F(x) ) \\
                                   & \equiv \exists x ( x \in D \land \lnot F(x) ) \\
                                   & \equiv \exists x \in D ( \lnot F(x) ) .
  \end{align*}
  次に式\eqref{eq:Dnotexists}を示す.
  \begin{align*}
    \lnot \exists x \in D(F(x)) & \equiv \lnot \exists x ( x \in D \land F(x) ) \\
                                & \equiv \forall x \lnot ( x \in D \land F(x) ) \\
                                & \equiv \forall x ( \lnot ( x \in D) \lor \lnot F(x) ) \\
                                & \equiv \forall x ( x \in D \to \lnot F(x) ) \\
                                & \equiv \forall x \in D ( \lnot F(x) ).
  \end{align*}
\item[\refque{que:tototo}] \mbox{} \\
  シークエント$A \to ( B \to C) \Longrightarrow A \land B \to C$の導出をする.
    \begin{enumerate}[1. ]
      \item $A \to ( B \to C) \Longrightarrow A \to (B \to C)$ 
             \quad [始式]
      \item $A \land B \Longrightarrow A \land B$ \quad [始式]
      \item $A \land B \Longrightarrow A$ \quad [2.から$\land$除去による]
      \item $A \land B ,  A \to ( B \to C) \Longrightarrow B \to C$
             \quad [$1., 3.$から$\to$除去による]
      \item $A \land B \Longrightarrow B$ \quad [2.から$\land$除去による]
      \item $A \land B ,  A \to ( B \to C) \Longrightarrow C$
             \quad [$4., 5.$から$\to$除去による]
      \item $A \to (B \to C ) \Longrightarrow A \land B \to C$
             \quad [6.から$\to$導入による]
    \end{enumerate}
    逆向きのシークエントも同様に導出できる.
\item[\refque{que:Peirce}] \mbox{}
  \begin{enumerate}[1. ]
    \item $(A \to B)  \to A \Longrightarrow (A \to B) \to A$ \quad [始式] 
    \item $\lnot A \Longrightarrow \lnot A$ \quad [始式(背理法で示す)]
    \item $A \Longrightarrow A$ \quad [始式]
    \item $\lnot A ,  A \Longrightarrow \curlywedge$ \quad [$2., 3.$から$\lnot$除去による]
    \item $\lnot A ,  A \Longrightarrow B$ \quad [4.から矛盾による]
    \item $\lnot A \Longrightarrow A \to B$ \quad [5.から$\to$導入による]
    \item $\lnot A,  (A \to B) \to A \Longrightarrow A$ \quad [$1., 6.$から$\to$除去による] 
    \item $\lnot A,  (A \to B) \to A \Longrightarrow \curlywedge$
           \quad [$2., 7.$から$\lnot$除去による]
    \item $(A \to B) \to A \Longrightarrow \lnot \lnot A$ \quad [8.から$\lnot$導入による]
    \item $(A \to B) \to A \Longrightarrow A$ \quad [9.から2重否定の除去による]
  \end{enumerate}
\item[\refque{que:togentei}] \mbox{} \\
  定理\ref{thm:genteito}と同様に示せる.
\item[\refque{que:genteilandor}] \mbox{} \\
  シークエント$\forall x (F(x) \lor A) \Longrightarrow \forall x F(x) \lor A$
  は背理法によって導出できる.あとは容易である.
\item[\refque{que:genteikoukan}] \mbox{} \\
  式\eqref{eq:forallkoukan}の証明は容易である.
  式\eqref{eq:existskoukan}の証明のうち,
  シークエント$\exists x \exists y F(x,y) \Longrightarrow \exists y \exists x F(x,y)$
  を導出しよう.
  \begin{enumerate}[1. ]
    \item $\exists x \exists y F(x,y) \Longrightarrow \exists x \exists y F(,y)$
           \quad [始式]
    \item $\exists y F(a, y) \Longrightarrow \exists y F(a,y)$
           \quad [始式($a$は新たな自由変数)]
    \item $F(a, b) \Longrightarrow F(a,b)$ \quad [始式($b$は新たな自由変数)]
    \item $F(a,b) \Longrightarrow \exists x F(x,b)$ \quad [3.から$\exists$導入による]
    \item $F(a,b) \Longrightarrow \exists y \exists x F(x,y)$
           \quad [4.から$\exists$導入による]
    \item $\exists y F(a,y) \Longrightarrow \exists y \exists x F(x,y)$
           \quad [$2., 5.$から$\exists$除去による]
         \item $\exists x \exists y F(x,y) \Longrightarrow \exists y \exists x F(x,y)$
           \quad [$1., 6.$から$\exists$除去による]
  \end{enumerate}
  逆向きのシークエントもまったく同様にして導出できることは明らかであろう.
\item[\refque{que:circdouti}] \mbox{} \\
  cut規則を繰り返し用いればよい.
\item[\refque{que:taisyousuii}] \mbox \\
  式\eqref{eq:taisyouritu}を導出する.
  \begin{enumerate}[1. ]
    \item $a =b \Longrightarrow a =a \to b =a$ \quad [置換法則]
    \item $\qquad \Longrightarrow a =a$ \quad [反射律]
    \item $a = b \Longrightarrow b =a$ \quad [$1., 2.$から$\to$除去による]
  \end{enumerate}
  次に,式\eqref{eq:suiiritu}を導出する.
  \begin{enumerate}[1. ]
    \item $a =b \land b=c \Longrightarrow a = b \land b =c$ \quad [始式]
    \item $b=a \Longrightarrow b=c \to a=c$ \quad [置換法則]
    \item $a = b \Longrightarrow b =a$ \quad [対称律]
    \item $a=b \land b=c \Longrightarrow a=b$ \quad [1.から$\land$除去による]
    \item $a=b \land b=c \Longrightarrow b=a$ \quad [$3., 4.$からcutによる]
    \item $a=b \land b=c \Longrightarrow b=c \to a=c$ \quad [$2., 5.$からcutによる]
    \item $a =b \land b=c \Longrightarrow b=c$ \quad [1.から$\land$除去による]
    \item $a=b \land b=c \Longrightarrow a=c$ \quad [$6., 7.$から$\to$除去による]
  \end{enumerate}
\item[\refque{que:2tusonzaihitei}] \mbox{} \\
  証明済みの関係式を用いれば容易であろう.
\item[\refque{que:tadahitotuexists}] \mbox{} \\
  式\eqref{eq:uniquetakadaka}の導出をする.ほかは省略する.
  \begin{enumerate}[1. ]
    \item $\exists x \forall y (F(y) \to y=x)
           \Longrightarrow \exists x \forall y (F(y) \to y=x)$ \quad [始式]
    \item $F(a) \land F(b) \Longrightarrow F(a) \land F(b) $ 
           \quad [始式($a,  b$は新たな自由変数)]
    \item $\forall y (F(y) \to y=c) \Longrightarrow \forall y (F(y) \to y=c)$
           \quad [始式($c$は新たな自由変数)]
    \item $F(a) \land F(b) \Longrightarrow F(a)$ \quad [2.から$\land$除去による]
    \item $\forall y (F(y ) \to y=c ) \Longrightarrow F(a) \to a =c $
           \quad [3.から$\forall$除去による]
    \item $\forall y (F(y) \to y=c) ,  F(a) \land F(b) \Longrightarrow a =c$
           \quad [$3., 4.$から$\to$除去による]
    \item $F(a) \land F(b) \Longrightarrow F(b)$ \quad [2.から$\land$除去による]
    \item $\forall y (F(y) \to y=c) \Longrightarrow F(b) \to b=c$
           \quad [3.から$\forall$除去による]
    \item $\forall y (F(y) \to y =c) ,  F(a) \land F(b) \Longrightarrow b =c$
           \quad [$7., 8.$から$\to$除去による]
    \item $b=c \Longrightarrow c =b$ \quad [対称律]
    \item $\forall y (F(y) \to y =c) ,  F(a) \land F(b) 
           \Longrightarrow c =b$ \quad [$9., 10.$からcutによる]
    \item $\forall y (F(y) \to y=c) \Longrightarrow a =c \land c=b$ 
           \quad [$6., 11.$から$\land$導入による]
    \item $a=c \land c=b \Longrightarrow a=b$ \quad [推移律]
    \item $\forall y (F(y) \to y=c) ,  F(a) \land F(b) 
           \Longrightarrow a =b$ \quad [$12., 13.$からcutよる]
    \item $\forall y (F(y) \to y =c) \Longrightarrow 
           F(a) \land F(b) \to a =b$ \quad [14.から$\to$導入による]
    \item $\forall y (F(y) \to y=c) \Longrightarrow 
           \forall y (F(a) \land F(y) \to a =y)$ \quad [15.から$\forall$導入による]
    \item $\forall y(F(y) \to y=c) \Longrightarrow 
           \forall x \forall y (F(x) \land F(y) \to x=y)$ \quad [16.から$\forall$導入による]
    \item $\exists x \forall y (F(y) \to y=x) \Longrightarrow 
           \forall x \forall y (F(x) \land F(y) \to x=y )$
           \quad [$1., 17.$から$\exists$除去による]
  \end{enumerate}
\end{description}
 % 演習問題の略解  
    }
  %
%
\begin{thebibliography}{99}
 \bibitem{kigouronri} 前原 昭二『記号論理入門』,2005年,日本評論社 \\
 これほどわかりやすく書かれた記号論理の解説にはなかなかお目にかかれるものではない.
 数学を学ぶのであればとりあえず購入していい本である.
 とはいえ,内容はそれほど豊富というわけではない
 (それでも本格的に足を突っ込むのでなければ十分)ので,
 そのあたりは他の本に頼る必要がある.
 \bibitem{suurironri} 鹿島 亮『数理論理学(現代基礎数学)』,2009年,朝倉書店 \\
 この資料のシークエント計算の解説に煮えきらなさを感じた方はこういう本を読むと良い.
 類書よりもたくさんの話題が取り上げられているお買い得パックである.
 \bibitem{kada} 嘉田 勝『論理と集合から始める数学の基礎』,2008年,日本評論社 \\
 ページ数が少ない割にたくさんの話題が解説されている.
 この資料よりも情報科学に必要な話題が多く取り上げられており,
 情報系の学生には自信を持っておすすめできる1冊である.
 \bibitem{tukuru} 戸田山 和久『論理学をつくる』,2000年,名古屋大学出版会 \\
 本当の意味で0からの入門書が欲しい人はこれ.
 fitch styleと呼ばれる流儀で演繹を行っている.
 とにかく分厚い.この資料の解説がいかに薄っぺらなものかがよくわかる本である.
 \bibitem{yosida} 吉田 夏彦『論理学』,1958年,培風館 \\
 この資料のシークエント計算の体系はこの本がもとになっている.
 現在入手するのは少々苦労するかもしれない.
 図書館の出番である.

\begin{comment}
 \bibitem{utidaset} 内田 伏一『集合と位相(数学シリーズ)』,1986年,裳華房 \\
 集合論の入門書としては定番の1冊.この資料の次に読むならこのあたりを読むと良い.
 \bibitem{utidatopo} 内田 伏一『位相入門』,1997年,裳華房 \\
 同著者による本.{\cite{utidaset}}を大学の講義で使いやすいよう書き直したものらしい.
 位相空間論にさっさとたどり着きたいなら{\cite{utidaset}}よりも
 こちらを手に取るべきかもしれない.
 ただ1点残念なのは論理記号の使い方がよろしくないことである.
 \bibitem{matsuzaka} 松坂 和夫『集合・位相入門』,1968年,岩波書店 \\
 とてもていねいな記述がなされており読みやすい.
 おそらく高校生でも問題なく読める.
 とはいえ,かなり古い本であり,現代の数学ではあまりホットではない話題も
 多く含まれている点には注意すべきである.
 \bibitem{30kouset} 志賀 浩二『集合への30講(数学30講シリーズ)』,1988年,朝倉書店 \\
 有名な30講シリーズの中の1冊.わかりやすいと評判らしい.
 \bibitem{30koutopo} 志賀 浩二『位相への30講(数学30講シリーズ)』,1988年,朝倉書店 \\
 \cite{30kouset}の続刊とも言うべき本.基本的に集合論と位相空間論はセットである.
 \bibitem{kunen} Kenneth Kunen(藤田 博司 訳)『集合論-独立性証明への案内』,2008年,日本評論社 \\
 公理的集合論に関する本.研究分野として集合論を学ぼうとするような人間が読む本である.
 とはいえ,そんな人間なら邦訳ではなく原著を読むべきである.
 一般人は図書館あたりでちょこっとチラ見する程度でいいかも.
\end{comment}

 \bibitem{latex} 奥村 晴彦・黒木 裕介『{\LaTeXe}美文書作成入門』,2017年,技術評論社 \\
 この本は数学の専門書ではなく,数学を学ぶのなら絶対に知らなければならない
 {\LaTeX}というフリーの組版システムの解説書である.
 まさかいつまでも手書きで文章を書くわけでもあるまいから,
 ちょっとくらい手を出してもバチは当たらないだろうと思う.
 とりあえず今現在使うのであればupLaTeXという種類の
 LaTeXを使うのがおすすめである.
\end{thebibliography}

    
 % 参考文献
{\footnotesize
\printindex[nidx] % 人名索引を出力
\printindex[widx] % 用語索引を出力
}
% \input{inputyou/okuduke/okuduke} % 奥付
%
%
\end{document}  
%%%%%%%%%%%%==========本文終わり===============%%%%%%%%%%%% 
 
        
