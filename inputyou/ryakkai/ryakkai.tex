\chapter{演習問題略解} \label{answer}
\begin{description}
  \item[\refque{chp:sequent.sec:ronri.que:singihantei}] \mbox{}
  \begin{enumerate}
   \item 偽
   \item 真
   \item 真
   \item 偽(反例は$x=3,  y=2$など)
   \item 偽
   \item 真
  \end{enumerate}
\item[\refque{chp:sequent.sec:hensuu.que:sokubakusingi}] \mbox{} 
  \begin{enumerate}
   \item 偽(反例は$x=1$など)
   \item 真
   \item 偽
   \item 真
   \item 真 ($x=3$とでもすればよい)
   \item 偽
  \end{enumerate}
\item[\refque{chp:sequent.sec:hituyoujubun.que:xhituyoujubun}] \mbox{} 
  \begin{enumerate}
   \item 正しくない
   \item 正しい
   \item 正しい
  \end{enumerate}
\item[\refque{que:kigoukaranihongo}] \mbox{} \\
  ここに挙げるのはあくまで一例である.
  細かい言い回しを考えれば,解答として適切なものはいくつも考えられるであろう.
  \begin{enumerate}
    \item  $F(x)$となる$x$がただひとつ存在する.
    \item 任意の$\varepsilon >0$に対して$\delta >0$が存在して,
      $0< \lvert x- a \rvert < \delta$を満たす任意の$x \in I$に対して
      $\lvert f(x) -A \rvert < \varepsilon$が成り立つ.
    \item 任意の$y \in Y$に対して$x \in X$が存在して,$y=f(x)$となる.
    \item $f(x_1 )=f(x_2) $を満たす任意の$x_1, x_2 \in X$に対して$x_1 = x_2$が成り立つ.
    \item 任意の$a,b >0$に対して自然数$N$で$Na >b$となるものがとれる.
  \end{enumerate}

\item[\refque{que:nihongokarakigou}] \mbox{} \\
  \begin{enumerate}
    \item $\forall x ( x \in A \to x \in B).$
      (「必ず」とあることを考えれば$x$には全称記号をつけるのが妥当であろう)
    \item $\forall \varepsilon >0 \exists N \in \mathbb{N} \forall n,m \in \mathbb{N}
      (n,m \geq N \to \lvert a_n - a_m \rvert < \varepsilon ).$
    \item $\forall x,y \in \mathbb{R} (x^2+y^2=1 \to \exists \theta \in \mathbb{R} 
      ( 0 \leq \theta <2 \pi \land (x= \cos \theta \land y= \sin \theta ))).$
    \item $\exists c \in \mathbb{R}(\forall x \in S ( x \leq c ) \land 
      \forall M \in \mathbb{R} (\forall x \in S (x \leq M) \to c \leq M)).$
  \end{enumerate}
\item[\refque{que:haityurituouyou}] \mbox{} \\  
  $\sqrt{2}^{\sqrt{2}}$が有理数の場合には$a = b= \sqrt{2}$とすればよい.
  $\sqrt{2}^{\sqrt{2}}$が有理数でない場合には,
  $\sqrt{2}^{\sqrt{2}}$は無理数であるから
  $a = \sqrt{2} ^{\sqrt{2}},  b= \sqrt{2}$とすればよい.

\item[\refque{que:meidaiketugouritu}] \mbox{} \\
  式\eqref{eq:lorketugouritu}の証明のうち,
  シークエント$A \lor ( B \lor C ) \Longrightarrow (A \lor B) \lor C$
  の導出をする(ほかのものは省略する).
  \begin{enumerate}[1. ]
    \item $A \lor ( B \lor C) \Longrightarrow A \lor (B \lor C)$
           \quad [始式]
    \item $A \Longrightarrow A$ \quad [始式]
    \item $A \Longrightarrow A \lor B$ \quad [2.から$\lor$導入による]
    \item $A \Longrightarrow (A \lor B ) \lor C$ \quad [3.から$\lor$導入による]
    \item $B \lor C \Longrightarrow B \lor C$ \quad [始式]
    \item $B \Longrightarrow B$ \quad [始式]
    \item $B \Longrightarrow A \lor B$ \quad [6.から$\lor$導入による]
    \item $B \Longrightarrow ( A \lor B) \lor C$ \quad [7.から$\lor$導入による]
    \item $C \Longrightarrow C$ \quad [始式]
    \item $C \Longrightarrow (A \lor B ) \lor C$ \quad [9.から$\lor$導入による]
    \item $B \lor C \Longrightarrow (A \lor B ) \lor C$
           \quad [$5., 8., 10.$から$\lor$除去による]
    \item $A \lor ( B \lor C) \Longrightarrow (A \lor B ) \lor C$
           \quad [$1., 4., 11.$から$\lor$除去による]
  \end{enumerate}

\item[\refque{que:Dnarabanot}] \mbox{} \\
  式\eqref{eq:Dnotall}を示す.
  \begin{align*}
    \lnot \forall x \in D ( F(x) ) & \equiv \lnot \forall x ( x \in D \to F(x) ) \\
                                   & \equiv \exists x ( x \in D \land \lnot F(x) ) \\
                                   & \equiv \exists x \in D ( \lnot F(x) ) .
  \end{align*}
  次に式\eqref{eq:Dnotexists}を示す.
  \begin{align*}
    \lnot \exists x \in D(F(x)) & \equiv \lnot \exists x ( x \in D \land F(x) ) \\
                                & \equiv \forall x \lnot ( x \in D \land F(x) ) \\
                                & \equiv \forall x ( \lnot ( x \in D) \lor \lnot F(x) ) \\
                                & \equiv \forall x ( x \in D \to \lnot F(x) ) \\
                                & \equiv \forall x \in D ( \lnot F(x) ).
  \end{align*}
\item[\refque{que:tototo}] \mbox{} \\
  シークエント$A \to ( B \to C) \Longrightarrow A \land B \to C$の導出をする.
    \begin{enumerate}[1. ]
      \item $A \to ( B \to C) \Longrightarrow A \to (B \to C)$ 
             \quad [始式]
      \item $A \land B \Longrightarrow A \land B$ \quad [始式]
      \item $A \land B \Longrightarrow A$ \quad [2.から$\land$除去による]
      \item $A \land B ,  A \to ( B \to C) \Longrightarrow B \to C$
             \quad [$1., 3.$から$\to$除去による]
      \item $A \land B \Longrightarrow B$ \quad [2.から$\land$除去による]
      \item $A \land B ,  A \to ( B \to C) \Longrightarrow C$
             \quad [$4., 5.$から$\to$除去による]
      \item $A \to (B \to C ) \Longrightarrow A \land B \to C$
             \quad [6.から$\to$導入による]
    \end{enumerate}
    逆向きのシークエントも同様に導出できる.
\item[\refque{que:Peirce}] \mbox{}
  \begin{enumerate}[1. ]
    \item $(A \to B)  \to A \Longrightarrow (A \to B) \to A$ \quad [始式] 
    \item $\lnot A \Longrightarrow \lnot A$ \quad [始式(背理法で示す)]
    \item $A \Longrightarrow A$ \quad [始式]
    \item $\lnot A ,  A \Longrightarrow \curlywedge$ \quad [$2., 3.$から$\lnot$除去による]
    \item $\lnot A ,  A \Longrightarrow B$ \quad [4.から矛盾による]
    \item $\lnot A \Longrightarrow A \to B$ \quad [5.から$\to$導入による]
    \item $\lnot A,  (A \to B) \to A \Longrightarrow A$ \quad [$1., 6.$から$\to$除去による] 
    \item $\lnot A,  (A \to B) \to A \Longrightarrow \curlywedge$
           \quad [$2., 7.$から$\lnot$除去による]
    \item $(A \to B) \to A \Longrightarrow \lnot \lnot A$ \quad [8.から$\lnot$導入による]
    \item $(A \to B) \to A \Longrightarrow A$ \quad [9.から2重否定の除去による]
  \end{enumerate}
\item[\refque{que:togentei}] \mbox{} \\
  定理\ref{thm:genteito}と同様に示せる.
\item[\refque{que:genteilandor}] \mbox{} \\
  シークエント$\forall x (F(x) \lor A) \Longrightarrow \forall x F(x) \lor A$
  は背理法によって導出できる.あとは容易である.
\item[\refque{que:genteikoukan}] \mbox{} \\
  式\eqref{eq:forallkoukan}の証明は容易である.
  式\eqref{eq:existskoukan}の証明のうち,
  シークエント$\exists x \exists y F(x,y) \Longrightarrow \exists y \exists x F(x,y)$
  を導出しよう.
  \begin{enumerate}[1. ]
    \item $\exists x \exists y F(x,y) \Longrightarrow \exists x \exists y F(,y)$
           \quad [始式]
    \item $\exists y F(a, y) \Longrightarrow \exists y F(a,y)$
           \quad [始式($a$は新たな自由変数)]
    \item $F(a, b) \Longrightarrow F(a,b)$ \quad [始式($b$は新たな自由変数)]
    \item $F(a,b) \Longrightarrow \exists x F(x,b)$ \quad [3.から$\exists$導入による]
    \item $F(a,b) \Longrightarrow \exists y \exists x F(x,y)$
           \quad [4.から$\exists$導入による]
    \item $\exists y F(a,y) \Longrightarrow \exists y \exists x F(x,y)$
           \quad [$2., 5.$から$\exists$除去による]
         \item $\exists x \exists y F(x,y) \Longrightarrow \exists y \exists x F(x,y)$
           \quad [$1., 6.$から$\exists$除去による]
  \end{enumerate}
  逆向きのシークエントもまったく同様にして導出できることは明らかであろう.
\item[\refque{que:circdouti}] \mbox{} \\
  cut規則を繰り返し用いればよい.
\item[\refque{que:taisyousuii}] \mbox \\
  式\eqref{eq:taisyouritu}を導出する.
  \begin{enumerate}[1. ]
    \item $a =b \Longrightarrow a =a \to b =a$ \quad [置換法則]
    \item $\qquad \Longrightarrow a =a$ \quad [反射律]
    \item $a = b \Longrightarrow b =a$ \quad [$1., 2.$から$\to$除去による]
  \end{enumerate}
  次に,式\eqref{eq:suiiritu}を導出する.
  \begin{enumerate}[1. ]
    \item $a =b \land b=c \Longrightarrow a = b \land b =c$ \quad [始式]
    \item $b=a \Longrightarrow b=c \to a=c$ \quad [置換法則]
    \item $a = b \Longrightarrow b =a$ \quad [対称律]
    \item $a=b \land b=c \Longrightarrow a=b$ \quad [1.から$\land$除去による]
    \item $a=b \land b=c \Longrightarrow b=a$ \quad [$3., 4.$からcutによる]
    \item $a=b \land b=c \Longrightarrow b=c \to a=c$ \quad [$2., 5.$からcutによる]
    \item $a =b \land b=c \Longrightarrow b=c$ \quad [1.から$\land$除去による]
    \item $a=b \land b=c \Longrightarrow a=c$ \quad [$6., 7.$から$\to$除去による]
  \end{enumerate}
\item[\refque{que:2tusonzaihitei}] \mbox{} \\
  証明済みの関係式を用いれば容易であろう.
\item[\refque{que:tadahitotuexists}] \mbox{} \\
  式\eqref{eq:uniquetakadaka}の導出をする.ほかは省略する.
  \begin{enumerate}[1. ]
    \item $\exists x \forall y (F(y) \to y=x)
           \Longrightarrow \exists x \forall y (F(y) \to y=x)$ \quad [始式]
    \item $F(a) \land F(b) \Longrightarrow F(a) \land F(b) $ 
           \quad [始式($a,  b$は新たな自由変数)]
    \item $\forall y (F(y) \to y=c) \Longrightarrow \forall y (F(y) \to y=c)$
           \quad [始式($c$は新たな自由変数)]
    \item $F(a) \land F(b) \Longrightarrow F(a)$ \quad [2.から$\land$除去による]
    \item $\forall y (F(y ) \to y=c ) \Longrightarrow F(a) \to a =c $
           \quad [3.から$\forall$除去による]
    \item $\forall y (F(y) \to y=c) ,  F(a) \land F(b) \Longrightarrow a =c$
           \quad [$3., 4.$から$\to$除去による]
    \item $F(a) \land F(b) \Longrightarrow F(b)$ \quad [2.から$\land$除去による]
    \item $\forall y (F(y) \to y=c) \Longrightarrow F(b) \to b=c$
           \quad [3.から$\forall$除去による]
    \item $\forall y (F(y) \to y =c) ,  F(a) \land F(b) \Longrightarrow b =c$
           \quad [$7., 8.$から$\to$除去による]
    \item $b=c \Longrightarrow c =b$ \quad [対称律]
    \item $\forall y (F(y) \to y =c) ,  F(a) \land F(b) 
           \Longrightarrow c =b$ \quad [$9., 10.$からcutによる]
    \item $\forall y (F(y) \to y=c) \Longrightarrow a =c \land c=b$ 
           \quad [$6., 11.$から$\land$導入による]
    \item $a=c \land c=b \Longrightarrow a=b$ \quad [推移律]
    \item $\forall y (F(y) \to y=c) ,  F(a) \land F(b) 
           \Longrightarrow a =b$ \quad [$12., 13.$からcutよる]
    \item $\forall y (F(y) \to y =c) \Longrightarrow 
           F(a) \land F(b) \to a =b$ \quad [14.から$\to$導入による]
    \item $\forall y (F(y) \to y=c) \Longrightarrow 
           \forall y (F(a) \land F(y) \to a =y)$ \quad [15.から$\forall$導入による]
    \item $\forall y(F(y) \to y=c) \Longrightarrow 
           \forall x \forall y (F(x) \land F(y) \to x=y)$ \quad [16.から$\forall$導入による]
    \item $\exists x \forall y (F(y) \to y=x) \Longrightarrow 
           \forall x \forall y (F(x) \land F(y) \to x=y )$
           \quad [$1., 17.$から$\exists$除去による]
  \end{enumerate}
\end{description}
