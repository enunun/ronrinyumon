\chapter{はじめに}
 この資料は,これから数学を本格的に学ぼうとする高校生,大学生を対象に,
 数学全体の基礎となっている「論理」や「推論」の感覚をつかんでもらうために書いたものである.
 
 本格的に数学を学ぼうとしたとき,
 まず第一に数学的な議論のフォーマットに体を慣らさなくてはならない.
 数学的な議論のフォーマットは,
 実際に数学を学ぶことを通して体得するのが普通である.
 そうして数学を学んでいくと,
 数学で使われる「推論」のなかには,直感的に明らかとは言い難いものが
 そこそこ多くあることに気がつくはずである.
 \index[nidx]{De Morgan@De Morgan(ド・モルガン)}
 De Morganの法則や対偶証明法,あるいは背理法などが挙げられるだろう.
 初めてそれらの論法が提示された際,
 ちょっとモヤモヤした気分になった人は私だけではあるまい.
 
 この資料では,まずは意味論的に論理記号を導入し,
 その後,自然演繹と呼ばれる演繹体系に基づいて
 これらの論法を正当化することを試みる.
 ただし,ここでは自然演繹そのものではなく,
 シークエント計算と呼ばれる手法を利用した自然演繹の体系を紹介する.
 個人的な意見ではあるが,自然演繹そのものよりもシークエント計算
 のほうがとっつきやすい.
 とはいえ,慣れるまではかなり苦労すると思われるから,
 根気よく学んでもらいたい.
 ページ数自体はあまり多くはないが,
 読み通すまでにはかなり時間がかかると思われる.

 なお,この資料は私1人で書いたものであり,
 校正も私自身が行っている.
 従って,普通の本よりも間違いや誤植が存在する可能性が高いということに
 留意されたい.
 もしも間違いや誤植を見つけた際には,以下に連絡していただけるとありがたい.

 \begin{itemize}
   \item TwitterID:@NOGUTAKULab
   \item E-mail:nogutakulab@gmail.com
 \end{itemize}

  
\begin{flushright}
  2018年1月 著者
\end{flushright}

 更新履歴
 \begin{itemize}
   \item 2018/1/10 公開
   \item 2018/1/11 誤植の修正,連絡先の追加
   \item 2018/4/5 「現代数学への展望」を追加,演習問題の追加,その他不適切な言い回しの修正
   \item 2018/7/30 レイアウトの修正,英訳の訂正,数学的構造の解説の追加など
 \end{itemize}
  
